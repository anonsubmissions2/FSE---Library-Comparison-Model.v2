\begin{practice}{\hypertarget{pracg1}{\pracg1{code}}}{\pracg1{}}
  \summary{In any project, you can adjust resources, scope, or schedule (the delivery triangle). When a project relies on volunteers, resources (contributors) are limited and cannot be adjusted.}
  \begin{pconcern}
    \item \qc{\hyperlink{problem3}{\problem3}}
    \item \qc{\hyperlink{problem4}{\problem4}}
  \end{pconcern}
  \solution{Adjust scope (quality or features) or schedule when project releases cannot be completed on schedule at the desired level of quality with the expected features.}
  \begin{related}
    \item \relspe{\hyperlink{pracg2}{\pracg2{code} \pracg2{}}}
    \item \relpre{\hyperlink{pracr11}{\pracr11{code} \pracr11{}}}
  \end{related}
  \challenges{Changing the schedule can have a negative impact on stakeholders relying on a posted schedule, on episodic contributors expecting a set cycle, and on habitual contributors who feel pressured to complete the work; changing the scope is often a better choice.}
  \observations{\sortinterviewees{DO, JL, AP}}
  %\proposals{\sortinterviewees{}}
\end{practice}




\begin{concern}{\hypertarget{problem11}{\problem11}} %% 5.C
When community members offer guidance to new participants, they often do so out of the expectation that the newcomers will become habitual contributors. However, a newcomer may not have this intention. Mismatched expectations about the commitment of a newcomer and the amount of effort that person will put into the project can lead to frustration and discouragement among habitual contributors involved in mentoring.
\end{concern}




