\begin{table}[]
    \centering
    \caption{Contributions and impactful research advances made in this study by the derived model and evaluated tools}
   
    \begin{tabular}{>{\raggedright}p{1.6cm}p{3.2cm}p{4.7cm}p{4cm}}%{llll}
        \toprule
        \textbf{Contribution} & \textbf{Summary} & \textbf{Research Advancement} & \textbf{Impact} \\
        \midrule
            Theoretical Model & We derived a library consumer behavior model with six dimensions of process, source, factors, conditions, barriers, and decision patterns. & Previous related research \cite{spinellis2019select, larios2020selecting, wasserman2017osspal,
    li2022exploring} focused on a partial view of library selection conditions or selection factors without any comprehensive relationship among all the dimensions.  & A comprehensive model can significantly enrich developer experience in the long run following the matured models in other domains such as marketing. \\ 
            Novel dimensions & We outlined the library adoption process, source information, barriers, and decision patterns. & We are aware of no previous study that addressed all these dimensions of process, sources, barriers, and decision patterns.  & 
            The process and decision patterns should guide new teams and organizations to establish their own robust policies.  \\ 
            Novel concepts & We identified 14 new conditions and 15 new library selection factors. & Previous research studies \cite{spinellis2019select, larios2020selecting, de2018library, de2018empirical, el2020libcomp, liu2021api, uddin2019understanding} found total nine influencing conditions and 13 library selection factors.  & Dependency and open source factors, just to name a few, are extremely important for legal perspectives that teams should be aware of. \\ \hline
            Library Selection Tool & From the derived behavior model, we conceptualized a library selection tool and evaluated its usefulness in the industry. & Previous tools supported review summarization \cite{lin2019pattern, uddin2019automatic, uddin2022empirical}, or collected library specific quantitative data \cite{de2018library, de2018empirical, el2020libcomp}, or compared multiple technologies \cite{huang2018tell, wang2020difftech, wang2021difftech, yan2022concept}. However, no study provided any concept on how the final decision of library selection can be made. & Practitioners appreciated our proposed tool and also suggested enhancements that can be used as groundwork for industry-grade useful tool development in future. \\ 
            
            BARD, ChatGPT evaluation for Library Selection & We experimented with chatbots for library selection and found the reliability concerns supported by the industry. & As per our knowledge, no previous studies have experimented with the feasibility of using ChatGPT/BARD for library selection. & 
            We also provided specific remedy areas such as online reference and multiple challenge techniques that can be further explored in future research. \\
            
        \bottomrule
    \end{tabular}
    \label{tab:contribution-summary}
\end{table}
