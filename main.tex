\documentclass[sigconf,review,anonymous, table]{acmart}
%\documentclass[sigconf,anonymous, table]{acmart}
%\documentclass[manuscript,screen, table]{acmart}
\usepackage{paralist}
\usepackage{enumitem,kantlipsum}
\usepackage{pgf-pie}
\usepackage{tikz}
\usepackage[most]{tcolorbox}
\usepackage{etoolbox}
\usepackage{fp}
\usepackage{subcaption}
\usepackage{mwe}
\usepackage{xspace}
\usepackage{eurosym}
\usepackage{mdframed,ifthen}
\usepackage{array}
\tcbuselibrary{listings,breakable}
\usepackage{placeins,newfloat}

\def\bf{\textbf}
\def\eq {Equation~}
\def\eqm {Eq~}
\def\eqs {Equations~}
\def\fig {Figure~}
\def\figs {Figures~}
\def\tbl {Table~}
\def\tbls {Tables~}
\def\ie{\textit{i.e.,}}
\def\eg{\textit{e.g.,}}
\def\sec {Section~}
\def\secs {Sections~}
\def\alg {Algorithm~}
\def\algs {Algorithms~}
\def\app {Appendix~}
\def\it{\textit}
\def\tr{\textrm}
\def\tt{\mct}
%\newcommand{\ib}[1]{{\textbf {\textit { #1}}}}
%\newcommand{\ts}[1]{{\textsc {{ #1}}}}
\newcommand{\mct}[1]{{\footnotesize {\texttt {#1}}}}

\newcommand{\qu}[1]{{\it{``#1''}}}
\newcommand{\api}[1]{{\sf{\texttt\small{#1}}}}
\newcommand{\callout}[1]{{\vspace{1mm}\noindent{\fbox{\parbox{0.97\columnwidth}{#1}}}\vspace{1mm}}}
\usepackage{paralist}
\usepackage{hyperref}
\usepackage{soul}


%% We use a different version in the main paper and in appendix!
\DeclareFloatingEnvironment[fileext=frm,placement={!ht},name=Frame]{nonIncrementingFloat}
\newenvironment{practice}[2]
    {
        \begin{center}
        \begin{nonIncrementingFloat}
        \begin{tcolorbox}[width=.48\textwidth, skin=enhanced jigsaw, breakable, arc=0pt, title={\textbf{Guiding Principle #1: #2}}]}
    {\end{tcolorbox}\end{nonIncrementingFloat}\end{center}
    }


\normalsize
\let\labelindent\relax
\usepackage{enumitem}

\newcommand{\nd}{\vspace{1mm}\noindent}
\usepackage{caption}
%\usepackage[toc,page]{appendix}
\usepackage{tikz}
\newcommand*\circled[1]{\tikz[baseline=(char.base)]{
            \node[shape=circle,draw,inner sep=1pt] (char) {#1};}}

 \lstset{
         language=Java,
         basicstyle=\scriptsize\ttfamily, % Standardschrift
         %numbers=left,               % Ort der Zeilennummern
         numberstyle=\tiny,          % Stil der Zeilennummern
         %stepnumber=2,               % Abstand zwischen den Zeilennummern
         numbersep=5pt,              % Abstand der Nummern zum Text
         tabsize=2,                  % Groesse von Tabs
        % extendedchars=true,         %
         breaklines=true,            % Zeilen werden Umgebrochen
%         keywordstyle=\color{black},
 %   	 frame=single,
 %        keywordstyle=[1]\textbf,    % Stil der Keywords
 %        keywordstyle=[2]\textbf,    %
 %        keywordstyle=[3]\textbf,    %
 %        keywordstyle=[4]\textbf,   \sqrt{\sqrt{}} %
         stringstyle=\color{white}\ttfamily, % Farbe der String
         showspaces=false,           % Leerzeichen anzeigen ?
         showtabs=false,             % Tabs anzeigen ?
         xleftmargin=17pt,
         framexleftmargin=17pt,
         framexrightmargin=5pt,
         framexbottommargin=4pt,
         %backgroundcolor=\color{lightgray},
         showstringspaces=false,      % Leerzeichen in Strings anzeigen ?
     %    escapeinside={\%*}{*)}
 }
\usepackage{rotating}
\usepackage{pdflscape}
\lstdefinestyle{inlinecode}{basicstyle={\ttfamily\scriptsize\bfseries}}
%\newcommand\code{\lstinline[style=inlinecode]}
%\newcommand{\urls}[1]{{\scriptsize\url{#1}}}
\usepackage{tcolorbox}
\newcommand{\emt}[1]{\emph{``#1''}}
\usepackage{paralist}
\usepackage[outercaption]{sidecap}    
%\usepackage{unicode-math}
\usepackage{amssymb}% http://ctan.org/pkg/amssymb
\usepackage{pifont}% http://ctan.org/pkg/pifont
\newcommand{\cmark}{\ding{51}}%
\newcommand{\xmark}{\ding{55}}%

\newcommand{\dc}[1]{\href{https://devrant.com/rants/#1/}{$R_{#1}$}}

\usepackage{enumitem}
\hypersetup{
    colorlinks=true,
    linkcolor=blue,
    filecolor=magenta,      
    urlcolor=cyan,
}
\newtcolorbox
{mybox}[2][]{colbacktitle=red!10!white,
colback=blue!10!white,coltitle=black!70!black,
title={#2},fonttitle=\bfseries,#1}

\usepackage{xcolor}
\definecolor{ao(english)}{rgb}{0.0, 0.5, 0.0}
\newcommand{\mybars}[3]{
  {\color{red}\rule{#1pt}{6pt}}
  {\color{yellow}\rule{#2pt}{6pt}}
  {\color{ao(english)}\rule{#3pt}{6pt}}
}
\newcommand{\mybarsbwv}[3]{
  {#1\%\color{black!90}\rule{#1pt}{6pt}}
  {#2\%\color{black!50}\rule{#2pt}{6pt}}
  {#3\%\color{black!20}\rule{#3pt}{6pt}}
}
\newcommand{\mybarsbw}[3]{
  {\color{black!90}\rule{#1pt}{6pt}}
  {\color{black!50}\rule{#2pt}{6pt}}
  {\color{black!20}\rule{#3pt}{6pt}}
}
\newcommand{\mybarsbwa}[3]{
  {#1\%Negative~\color{black!90}\rule{#1pt}{6pt}}\\
  {#2\%Neutral~\color{black!50}\rule{#2pt}{6pt}}\\
  {#3\%Positive~\color{black!20}\rule{#3pt}{6pt}}
}

%\newcommand{\rev}[1]{\textcolor{blue}{ #1}}
\newcommand{\rev}[1]{#1}
\newcommand{\newrev}[1]{\textcolor{blue}{ #1}}
\newcommand{\as}[1]{{\textcolor{magenta}{\sf{\texttt\small{#1}}}}}

\newcommand{\code}[1]{{\fontfamily{lmdh}\selectfont{\small\textsl{#1}}\normalfont}} 
\newcommand{\ab}[1]{{\color{red}{Ann: #1}}}
\newcommand{\numInterviews}{24\xspace}
\newcommand{\todo}[1]{{\color{red}{\textbf{#1}}}}

 %% Ann: I hate the way \qw looks so I've replaced \qw with this new command, \qqw. If the consensus is to return to the look of \qw, this command can be redefined.
 \newcommand{\qqw}[2]{{{\fontfamily{lmdh}\selectfont{\small\textsl{#2}}\normalfont}} (#1)\xspace} %% Usage: \qqw{F}{stability}
 \newcommand{\qww}[1]{(#1)\xspace} %% Usage: \qww{F}



\newcommand{\gias}[1]{\textcolor{red}{{[Gias: #1]}}}

% Used to format quotations. Usage:
% Attributed quote: \quotebox{P4}{That's annoying.}
% Unattributed quote: \quotebox{}{I've heard the comment.}
\newenvironment{smallquote}%
{\list{}{\leftmargin=0.15in\rightmargin=0.15in}\item[]}%
  {\endlist}
\newcommand{\quotebox}[2]{\begin{smallquote}\textit{``#2''}\ifthenelse{\equal{#1}{}}{}{ \mbox{-}~#1}\end{smallquote}}

\newenvironment{observation}[2]{
        \begin{mdframed}[
                frametitle={\colorbox{white}{\space #1}},
                innertopmargin=0pt,
                frametitleaboveskip=-\ht\strutbox,
                frametitlealignment=\raggedright,
                nobreak=true,
                skipabove=10pt,
                skipbelow=10pt,
        ]%
        \label{#2}}{\end{mdframed}}



\newcounter{recommendation}[section]
\renewcommand{\therecommendation}{\arabic{recommendation}}

\newenvironment{recommendation}[1]{
        %\refstepcounter{observation}
        \addtocounter{recommendation}{1}
        \begin{mdframed}[
                frametitle={\colorbox{white}{\space Recommendation \therecommendation\space}},
                innertopmargin=0pt,
                frametitleaboveskip=-\ht\strutbox,
                frametitlealignment=\raggedright,
                nobreak=true,
                skipabove=10pt,
                skipbelow=10pt,
        ]%
        \label{#1}}{\end{mdframed}}


\def\nonzero#1#2{%
    \ifnum #1 > 0
      #1#2
    \fi
}


% \newcommand{\horizontalbars}[4]{
% {{\color{black}\rule{#1pt}{4pt}}  \nonzero{#1}{C}}
% {{\color{black!50}\rule{#2pt}{4pt}}  \nonzero{#2}{F}}
% {{\color{black!20}\rule{#3pt}{4pt}}  \nonzero{#3}{P}}
% {{\color{black!40}\rule{#4pt}{4pt}} \nonzero{#4}{S}}

% }

% \newcommand{\horizontalbars}[5]{
% {{\color{black}\rule{#1pt}{4pt}}  \nonzero{#1}{NA}}
% {{\color{red}\rule{#2pt}{4pt}}  \nonzero{#2}{AS}}
% {{\color{green}\rule{#3pt}{4pt}}  \nonzero{#3}{EU}}
% {{\color{blue}\rule{#4pt}{4pt}} \nonzero{#4}{AU}}
% {{\color{black!50}\rule{#5pt}{4pt}} \nonzero{#5}{SA}}
% }


\newcommand{\horizontalbars}[5]{
{{\color{black}\rule{\fpeval{5*(#1)}pt}{4pt}}\nonzero{#1}{}}{{\color{red}\rule{\fpeval{5*(#2)}pt}{4pt}}\nonzero{#2}{}}{{\color{green}\rule{\fpeval{5*(#3)}pt}{4pt}}\nonzero{#3}{}}{{\color{blue}\rule{\fpeval{5*(#4)}pt}{4pt}}\nonzero{#4}{}}{{\color{black!50}\rule{\fpeval{5*(#5)}pt}{4pt}}\nonzero{#5}{}}
}

%\newcommand{\mybarsbwa}[4]{}
% \\   {C\color{black!90}\rule{#1pt}{6pt}}#1 {P\color{black!70}\rule{#2pt}{6pt}}#2 {F\color{black!40}\rule{#3pt}{6pt}}#3 {S\color{black!20}\rule{#4pt}{6pt}}#4
% }


%\newcommand{\qq}[2]{} % remove all quotes
%\newcommand{\qq}[2]{\textit{"#1"}$_{#2}$} % inline quotes
%\newcommand{\qi}[2]{\textit{"#1"}$_{#2}$} % inline quotes
%\newcommand{\qq}[2]{\begin{quote}"#1"$_{#2}$\end{quote}} % stylized quotes

\newcommand{\qi}[2]{\quotebox{#2}{#1}}
\newcommand{\qq}[2]{\quotebox{#2}{#1}} 
\newcommand{\qqi}[2]{\textit{"#1"} - {#2}} % inline quotes




%This is an apple {\def\svgwidth{2cm}\input{name.pdf_tex}} and more text
%https://tex.stackexchange.com/questions/374192/how-to-use-figures-as-inline-images
%https://tex.stackexchange.com/questions/313927/tikz-picture-inline
%https://tex.stackexchange.com/questions/7032/good-way-to-make-textcircled-numbers
\newcommand{\qw}[2]{\textbf{#2}\tikz[baseline=(char.base)]{
    \node[shape=circle,fill=blue!20,inner sep=.5pt] (char){#1};}}

\newcommand{\tc}[0]{\textcolor{green}{ [add citation]}}

% following command is used to highlight text/numbers which can be changed after all the inerview/data collection is done. 
\newcommand{\td}[1]{\textcolor{blue}{(#1)}}

\newcommand{\autourfill}[1]{\tikz[baseline=(X.base)]\node [draw=blue,fill=blue!40,semithick,rectangle,inner sep=2pt, rounded corners=3pt] (X) {#1};}

\newcommand{\autouroutline}[1]{\tikz[baseline=(X.base)]\node [draw=blue,fill=white,semithick,rectangle,inner sep=2pt, rounded corners=3pt] (X) {#1};}

\newcommand{\autourbox}[1]{\tikz[baseline=(X.base)]\node [draw=black!70,fill=white,semithick,rectangle,inner sep=2pt, rounded corners=3pt] (X) {#1};}

\newcommand{\autourhighlight}[1]{\tikz[baseline=(X.base)]\node [draw=none,fill=red!20,semithick,rectangle,inner sep=2pt, rounded corners=3pt](X){#1};}




\newcommand\actor[1]{\textbf{Actor:}  #1\vspace*{.5em}\\} 
\newcommand\condition[1]{\textbf{Condition:}  #1\vspace*{.5em}\\} 
\newcommand\concern[1]{\textbf{Concern:} #1\vspace*{.5em}\\}
\newcommand\solution[1]{\textbf{Solution:} #1\vspace*{.5em}\\ }
\newcommand\consideration[1]{ \textbf{Consideration:} #1 }
\newcommand\steps[1]{\vspace*{.5em}\\\textbf{Steps:} #1}
\newcommand\example[2]{\vspace*{.5em}\\\textbf{Example Data:} \qqi{#1}{#2}}

%\newcommand{minaoar}[1]{\textcolor{blue}{Minaoar: #1}}

%\newcommand\minaoar[1]{\textcolor{blue}{#1}}
\newcommand{\minaoar}[1]{\textcolor{blue}{{[Minaoar]: #1}}}
\AtBeginDocument{%
  \providecommand\BibTeX{{%
    \normalfont B\kern-0.5em{\scshape i\kern-0.25em b}\kern-0.8em\TeX}}}

\setcopyright{acmcopyright}
\copyrightyear{2023}
\acmYear{2023}
\acmDOI{XXXXXXX.XXXXXXX}

%% These commands are for a PROCEEDINGS abstract or paper.
\acmConference[ESEC/FSE 2023]{The 31st ACM Joint European Software Engineering Conference and Symposium on the Foundations of Software Engineering}{11 - 17 November, 2023}{San Francisco, USA}

%
%  Uncomment \acmBooktitle if th title of the proceedings is different
%  from ``Proceedings of ...''!
%
%\acmBooktitle{Woodstock '18: ACM Symposium on Neural Gaze Detection,
%  June 03--05, 2018, Woodstock, NY} 
\acmPrice{15.00}
\acmISBN{978-1-4503-XXXX-X/18/06}

\begin{document}

% \title{Guiding Principles for the Software Library Adoption Lifecycle: Perspectives from Industry}

%\title{A Model of Software Library Adoption in the Industry}
\title{Software Library Adoption Process and Principles in the Industry}

%\title{Promises and Perils of the Software Library Adoption Life Cycle}
%\title{Processes and Principles Guiding Software Library Adoption in the Industry: Perspectives from Industrial Developers}
%\title{The Life Cycle and Principles of Software Library Adoption: Perspectives from Industrial Developers}

%\title{How Industrial Developers Compare \& Adopt Software Libraries}
%\title{To Buy or Not to Buy: How Developers Buy Third Party Libraries}
%\title{Third Party Libraries: Reusable Wheel that Requires Continuous Tire Pressure Monitoring}
%\title{Choosing Third Party Libraries: A Common Necessity with Maintenance Baggage}
%\title{Choosing Third Party Libraries: A Vehicle with Risk of Brake Failure}

\author{Minaoar Tanzil}
%\authornote{Both authors contributed equally to this research.}
\email{minaoar.tanzil@ucalgary.ca}
%\orcid{1234-5678-9012}
\author{Gias Uddin}
%\authornotemark[1]
\email{gias.uddin@ucalgary.ca}
\author{Ann Barcomb}
%\authornotemark[1]
\email{ann.barcomb@ucalgary.ca}
\affiliation{%
  \institution{University of Calgary}
  \streetaddress{P.O. Box 1212}
  \city{Calgary}
  \state{Alberta}
  \country{Canada}
  \postcode{T2N 1N4}
}


%%
%% By default, the full list of authors will be used in the page
%% headers. Often, this list is too long, and will overlap
%% other information printed in the page headers. This command allows
%% the author to define a more concise list
%% of authors' names for this purpose.
\renewcommand{\shortauthors}{Tanzil et al.}

%%
%% The abstract is a short summary of the work to be presented in the
%% article.
\begin{abstract}
Modern day rapid software development is often facilitated by the reuse of third-party software libraries. Software library adoption includes the selection as well as the maintenance of a library. Various factors and conditions may influence the adoption of a software library in a company. However, while literature offers insights on what factors could influence the selection of library, we are not aware of any study that offered insights into the overall process and principles that companies follow to adopt a software library. In this paper, our research explores how the conditions and factors of the environment influence developer decisions in library selection and maintenance. Using Straussian grounded theory, we conducted \numInterviews semi-structured interviews. The result is a theoretical framework of a library adoption model, six guiding principles of library adoption, and six recommendations for developers and organizations. Our work lays the groundwork for the development of a comparative library analysis tool to enable efficient decisions about libraries.
%consisting of 5 steps, 5 major information sources, 53 selection factors and conditions and 6 overarching guiding principles. The library adoption model can be used for the development of a comparative library analysis tool to enable faster and more accurate decisions among developers.
\end{abstract}

%%
%% The code below is generated by the tool at http://dl.acm.org/ccs.cfm.
%% Please copy and paste the code instead of the example below.
%%
\begin{CCSXML}
<ccs2012>
   <concept>
       <concept_id>10011007.10011006.10011072</concept_id>
       <concept_desc>Software and its engineering~Software libraries and repositories</concept_desc>
       <concept_significance>500</concept_significance>
       </concept>
   <concept>
       <concept_id>10003120.10003130.10011762</concept_id>
       <concept_desc>Human-centered computing~Empirical studies in collaborative and social computing</concept_desc>
       <concept_significance>300</concept_significance>
       </concept>
 </ccs2012>
\end{CCSXML}

\ccsdesc[500]{Software and its engineering~Software libraries and repositories}
\ccsdesc[300]{Human-centered computing~Empirical studies in collaborative and social computing}

%%
%% Keywords. The author(s) should pick words that accurately describe
%% the work being presented. Separate the keywords with commas.
\keywords{third-party, software library, adoption process, guiding principle}
%%
%% This command processes the author and affiliation and title
%% information and builds the first part of the formatted document.
\maketitle
\balance
\section{Introduction}
In 2011, Marc Andersson famously noted that \textit{"Software is eating the world"} \cite{website:eat-world}. Indeed, software has now become the enabler of almost every tasks we do or services we offer/consume. With growing demands of tools and techniques, software companies seek to build and deploy their products quickly and efficiently. Modern day rapid software development is often facilitated by the reuse of third-party software libraries or APIs (Application Programming Interfaces). The libraries can be open-source or proprietary. Nevertheless, software/IT companies benefit from the adoption of third-party software libraries and often prefer the reuse of a library over the of re-implementation of a feature from scratch~\cite{uddin2017opiner}. For example, according to a 2017 report by the European Commission, the use of open source software saves European companies approximately \euro456 billion per year through increased productivity, efficiency, and direct cost savings \cite{eu2017economic}. These advantages come from standardization and reductions in vulnerability, development time, and expertise. However, such benefits do not always accrue; Spinellis outlines many potential concerns of relying on an open source library, among them: the project might not be maintained, the code may not be of high quality, the license may not allow the desired use, and the documentation may be poor \cite{spinellis2019select}.

The process of selecting a library is not simple, as it requires balancing a number of considerations, both those inherent to the library itself and those relating to the context in which it will be used. Developers or the deciding team in a company must ask questions like whether the library will accommodate the company's expansion plans and what the full legal implications are of using the library (including potentially hidden obligations) 
\cite{spinellis2019select,wolter:2022:open}. Factors such as the fit between functional capability of available software and the acceptance (or rejection) of the technology by peers may also influence the selection decision \cite{dishaw1998supporting,eckhardt2009influences}. % Dishaw: Fit is the matching of the capabilities of the tech- nology to the demands of the task.

While open source software represents a large proportion of the external libraries employed, companies also routinely use proprietary libraries, some of which may be part of a long supply chain which also involves open source software \cite{harutyunyan:2018:understanding}. All of the considerations of open source software still apply: the company needs to consider if the library will continue to be developed, is a good fit for the task, and so on. Further, while the code for open source libraries is freely available, there may still be financial considerations such as switching costs, support, and access to extended (proprietary) features \cite{dahlander2006business}. Therefore, while certain factors may be of greater or lesser importance to a company when evaluating a particular library, the possible considerations are not substantially different for proprietary and open source libraries. 

Indeed, it is crucial to pick the right software library in a large-scale software system, given that the replacement of a library because a new library can induce breaking changes into the system that take significant time to address. At the same time, the adoption of a new software library can be a time consuming and resource intensive process in a large software company, e.g., legal team needs to check for compatible licenses, product team needs to determine the best candidate out of the available libraries, etc. It is thus important to understand the decision making processes software companies follow to select and adopt a new software library and the factors and principles they consider during the process. 

The selection factors of software libraries have been the subject of several studies. In particular, technical criteria related to the selection of open source tools are studied and cataloged extensively \cite{wasserman2017osspal, li2022exploring, larios2020selecting, huang2018tell, wang2020difftech, wang2021difftech, uddin2017automatic, uddin2017opiner, de2018library, de2018empirical, el2020libcomp, yan2022concept, liu2021api, uddin2019understanding, larios2020selecting}. However, the adoption of a software library in industry can involve both the selection and the subsequent maintenance of the library. As such, software library adoption in the industry is a multi-faceted process that can encapsulate an end-to-end process, where a given step can be influenced by diverse factors and principles.  A holistic understanding of the adoption process is missing in the literature. 

In this paper, we conducted 24 semi-structured interviews of industrial developers to understand the adoption process of third-party software in a company. The outcome is a theoretical framework of software library adoption model that consists of a catalog of steps, conditions, factors, and principles related to software library adoption in the industry. By conditions, we mean environmental characteristics which are independent of the library under consideration. By contrast, by factors we mean characteristics of the library itself. We answer three research questions to describe our derived software library adoption model (RQ).

\nd\bf{RQ1: What are the major steps developers follow in the industry during their adoption of a software library?} We found that developers go through five iterative steps of search, compare, review, integrate and maintain for library adoption. \fig\ref{fig:adoption-process} shows how the five steps can support the selection and adoption process of software libraries. During this process, software developers/teams collect library related information from five major information sources (e.g. online forums, colleagues, etc.). To the best of our knowledge, ours is the first study that presents an end-to-end adoption process like \fig\ref{fig:adoption-process} (explained further in \sec\ref{sec:phases}).
%\minaoar{should we also note that this adoption process is first in literature for software libraries? For RQ2 and RQ3, we compared with existing literature.}
\begin{figure}[h]
    \centering
    \includegraphics[scale=.6]{images/adoption-process.pdf}
    \caption{Software library adoption process}
    \label{fig:adoption-process}
\end{figure}
% As we understand the adoption lifecyle, we realized that there are several conditions and factors that can influence the adoption lifecyle. We thus sought to answer the following research question.

\nd\bf{RQ2: What conditions and technical factors influence the library adoption process?}
We found that developers have to consider 23 types of conditions (e.g., environmental, organizational, etc.) and 28 types of factors (e.g., technical, commercial)  during the library adoption process. While 22 of these conditions and factors are previously reported in the literature  (e.g., popularity), 29 others were observed the first time in our study (e.g., benchmarking).
% As developers follow the adoption process and consider the influential conditions and factors, we found that their final decisions on whether and how to adopt a library can be encoded into several guiding principles. As such, we sought to answer the final research question as follows. 

\nd\textbf{RQ3: What guiding principles do developers follow during the library adoption process?} We found that developers follow 6 guiding principles (e.g., maximize flexibility during adoption) to achieve the optimum benefits from libraries and manage the burdens associated with those libraries. To the best of our knowledge, ours is the first study that reports and catalogs the principles that could guide the adoption of software libraries.

Based on the derived library adoption model, we offer six recommendations that software companies and developers can follow to guide their library adoption process. 

% To address this gap in the literature, we tried to discover the end to end library adoption life cycle so that we could not only understand how developers select and maintain libraries, but also realize their core motivations that guides developer throughout their library adoption journey. In this study, we considered the process through the lens of consumer buyer behavior, where the product-specific factors determine what the consumer will select, in the context of consumer-specific concerns, or `influences' \cite{kotler2014principles}. By using this approach, we have sought to explain the complexity of the adoption of software libraries, a decision which has often been considered as primarily driven by technical considerations. Our work was driven by the following questions:
% % In order to better understand the complete process by which library selection decisions are made, we posed the following questions:
% \begin{itemize}
% \item \textbf{RQ1}: What are the major steps developers follow during their adoption of a software library?
% \item \textbf{RQ2}: What conditions and technical factors influence the library adoption process?
% \item \textbf{RQ3}: What guiding principles do developers follow in library adoption?
% \end{itemize}


% \textbf{Summary of Contribution} 
% Novel contribution: \newline
% A. Library Adoption Life Cycle \newline
% B. Library Adoption Guiding Principles
% \todo{@Someone - spell out the contributions. Then make sure that they are reflected in the abstract.}
% \@todo{Culminating in X recommendations for industry}

\nd\bf{Replication Package \cite{website:replication-package}} contains anonymized transcribed interviews, study protocol, codebook and scripts for the results.
% Our replication package contains the interview script, invitation emails, consent form, REB approval, final codebook, assessment of grounded theory evaluation criteria, and an Appendix \cite{website:replication-package}. In order to preserve the anonymity of our participants and to adhere to our ethics approval, we do not share a list of interview participants or the interview transcripts. 



% The rest of this paper is organized as follows. First, we put the research in the context of related work. Next, we explain the methodology and the results. This is followed by a discussion and a consideration of the limitations of the research. Finally, we summarize the key findings.




% \section{Background and Related Work}
% %\begin{table*}[]
    \centering
    \caption{Research works related with software, technology, and library selection and comparison tools and factors}
    \begin{tabular}{p{3cm}p{4cm}p{10cm}}
    \toprule
    \textbf{Output} & \textbf{Study} & \textbf{Summary} \\ 
    \midrule
    Software Evaluation Tool & Open source software evaluation factors and tool \cite{wasserman2017osspal,
    li2022exploring} & Provides evaluation criteria (factors) for open source software (not library component). Identified 8 factors and 74 sub-factors. \\ 
    Technology Comparison Tool & \textbf{DiffTech:} technology comparison tool \cite{huang2018tell, wang2020difftech, wang2021difftech} & The tool mines comparative opinions in Stack Overflow about pairs of technologies without any predefined selection factors. \\ 
    \textbf{Library factors} related data & \textbf{LibComp:} Quantitative data collection on library specific factors \cite{de2018library, de2018empirical, el2020libcomp} & Proposed 9 quantitative factors of libraries and developed an online tool and an IDE plugin to display library specific information. \\ 
    \textbf{Library factors} related opinion & \textbf{Opiner:} Opinion summarization of library specific factors \cite{uddin2017automatic, uddin2017opiner} & Developed an API opinion summarization tool by crawling online forums contents regarding posts related with different APIs and their aspects. \\ 
    Opinion Mining Technique & Opinion mining and sentiment analysis techniques \cite{lin2019pattern, uddin2019automatic, uddin2022empirical} & Prepared benchmark dataset for opinions, and proposed pattern based or machine learning based techniques. \\ 
    Library Related Needs & Developer needs identification from Stack Overflow \cite{liu2021api, uddin2019understanding} & Identified 8 API related needs expressed in Stack Overflow and reported library selection factors from developer survey. \\ 
    Library Comparison UI & User interface for library comparison \cite{yan2022concept} & The interactive user interface presents concept (aspect) annotated code examples side by side for comparing multiple libraries. \\ 
    \textbf{Library Factors} for selection criteria & Library selection factors \cite{spinellis2019select, larios2020selecting} & Presented library selection factors from experience and from practitioners' survey and interview. \\ 
    
    
    \bottomrule
    \end{tabular}
    \label{tab:related-works-summary}
\end{table*} % AB: Save this for the tool paper

% We first describe the buyer behavior models which we used as the lens for viewing software library adoption. Next, we review two strands of literature which relate to our research questions. 



% \subsection{Background}
% \subsubsection{Buyer Behavior Models}
% We use buyer behavior as the lens through which we examine software library adoption decisions.
% In contrast to technology adoption models, consumer and business buyer behavior model drawn from marketing theories provide more in-depth insights into how a consumer or a business organization makes a decision to procure a product. In their seminal textbook, Kotler and Armstrong defined consumer/organization specific concerns as `influences' and  separated these from product-specific attributes (which we will henceforth refer to as factors, consistent with the software engineering literature) while explaining the influences in buying process \cite{kotler2014principles}. Moreover the product-specific factors are also a foundational element of the Fishbein Multiattribute Model which calculates the weighted average of all product-specific factors to define which product the consumer will select \cite{fishbein1967attitude}. Since it was introduced almost 60 years ago, this model has been ``extensively used by consumer researchers'' \cite{blackwell2001consumer}. %The segregation between product (here library) attributes and decision influences was established in marketing theory because of the elaborate analysis of the buying decision process. 
% In addition to influential conditions, and important product factors, the buyer behavior models also provided steps of buying decision process and actors involved or influencing the product adoption. Though these models provide more holistic approach compared to technology adoption models, there is no study how such models are applicable to technology adoption specifically, library adoption process.

% By contrast, research on library selection has focused almost entirely on technological factors, and has not looked holistically at the entire library selection process while distinguishing product-specific factors from influences.
\section{Related Work}
  In section~\ref{sec:tech-adoption}, we look at the steps of technology adoption, which relates to our first research question. In \sec\ref{sec:lit:processes}, we approach our second and third research questions by discussing the factors used in evaluation and the formal processes which have been developed to facilitate the decision, usually at an organizational level. 
  %Section~\ref{sec:lit:automation} covers efforts that have been made to automate gathering information to facilitate software selection decisions.
\subsection{Steps of Technology Adoption} %% Background for RQ1
\label{sec:tech-adoption}
The process by which a tool comes to be adopted within an open source software community was found to consist of several phases: knowledge, individual persuasion, individual decision, individual implementation, organizational adaptation, and organizational acceptance \cite{krafft:2016:free}. In the knowledge phase, for example, sedimentation describes a technology gaining recognition to the point that it is considered ready for adoption. Marketing, or the active promotion of the tool or technique to generate excitement, simulates developers to evaluate the project's utility. The final knowledge phase, peercolation, explains how information spreads between peers and how this information from trusted sources leads to favored treatment of the technology.


% Considering all related works, there has been a vacuum and necessity for understanding the complete process and principles of library adoption (not only selection, rather than maintenance as well) and its process  which is performed in this study. 

In marketing, psychology, and technology literature, a number of technology adoption models have been proposed. Fishbein and Azjen's Theory of Reasoned Action (TRA) \cite{flanders1975belief-tra} explaining consumer's belief, attitude, intention, and behavior has become foundational base for investigating personal technology usage \cite{taherdoost2018-adoption-models}. Two notable derivation of TRA model are Theory of Planned Behavior (TPB) \cite{ajzen1991-tpb} which added perceived behavioral control and Technology Acceptance Model (TAM) \cite{davis1985tam, davis1989-tam-usefulness} which considered perceived usefulness, perceived ease of use, and attitude toward use for technology adoption by individual consumers. While these models explained consumer behavior to adopt or accept a technology product, TOE Framework by Tornatzky et al. \cite{tornatzky1990processes-toe} describes how the adopting and implementing technology in organizations is influenced by Technological, Organizational, and Environmental (TOE) conditions. Whereas TOE framework provides insights how organizational adoptions could be influenced, this framework has been criticized to be too generic \cite{zhu2005post-toe-critic} and not sufficient enough for explaining software library adoption process.



\subsection{Technology Selection Factors and Processes}\label{sec:lit:processes}

Perhaps the most obvious factors involved in the selection of libraries relate the the needs of developers and the utility offered by the library. Historic factors which limited software reuse were limitations in the capacity of retrieval technologies to return relevant results and the difficulty of evaluating components \cite{hummel2008code}. Retrieval is no longer a serious concern, as modern software ecosystems have developed that make it easy for developers to incorporate libraries with searches built in to IDEs. However, the evaluation of components remains difficult, especially when the process can be hidden to the extent that malicious code can be hidden in dependency chains of popular packages \cite{wyss2022wolf}. Developers seek out API reviews in order to discover information related to their specific development needs and to compensate for shortcomings in the official documentation \cite{uddin2019understanding}. Liu et al. created a taxonomy of the API elements required by developers based on an analysis of Stack Overflow posts, identifying categories such as non-functional improvement, error handling, and API usage learning \cite{liu2021api}. Developers make use of a ``combination of code examples and opinions about an API as a form of documentation,'' relying on the positive and negative views in order to assess the quality of the API \cite{uddin2019understanding}. This is complicated by concerns about the trustworthiness, relevance, and recency of the opinions. % Also uddin2019understanding

While the aforementioned studies have focused largely on technical aspects of library selection, Larios-Vargas et al. conducted the most comprehensive study of factors associated with library selection, identifying 26 through semi-structured interviews, which were subsequently validated through a survey of 116 developers \cite{larios2020selecting}. The factors were categorized as technical, human, and economic. In contrast to our study, they did not investigate the conditions under which each factor is relevant in the selection process. Furthermore, the study did not draw a clear distinction between factors that are directly related to the libraries, and those which stem from the external environment, such as company culture, company management, and type of industry. These concerns undoubtedly influence the selection process, but the lack of segregation of library-specific factors and conditions, together with the lack of guidance on the when factors should be considered limits the extent to which the study can support the decision-making process.

% Shorter version of paragraph.
Companies trying to select the most appropriate library for a task have been provided with processes - sometimes with tool support - which strive to help them compare factors based on their internal conditions, such as Qualification and Selection Open Source (QSOS) and Open Business Readiness Rating (OpenBRR) \cite{deprez2008comparing, semeteys2008method, wasserman2017osspal}. Li et al. also identified eight key evaluation factors in considering open source software applications (e.g., community and adoption, development process, etc.) \cite{li2022exploring}.  Cruz, Wieland and Ziegler proposed a process which considers the scenario for adopting, the requirements, the interpretation of collected information, and an investigation of the criteria \cite{cruz2006evaluation}. For each scenario, the authors identify the functional, technical, organisational, economical, and political requirements associated with it. While these attempts to codify the decision process do address human and technical factors, they are primarily focused on the selection of applications, not libraries.
%% Long version of paragraph below.
% There have been multiple processes proposed for companies trying to select the most appropriate open source software for a task. Although the emphasis is typically on products, libraries are occasionally mentioned and indeed, there is nothing in the processes which would rule out their use on components. Two such processes, both of which are based on a comparison of factors based on conditions, are Qualification and Selection Open Source (QSOS) and Open Business Readiness Rating (OpenBRR). \cite{deprez2008comparing}. In QSOS, the user begins with a list of projects which seem to fit the overall requirements, evaluates them according to the criteria given by QSOS (such as intrinsic durability, technical adaptability, and documentation), adjusts the importance of each criterion based on their conditions, and then makes a decision \cite{semeteys2008method,deprez2008comparing}. OpenBRR reduces a long list of projects to a short list of candidates based on the criticality of the system and context-specific criteria such as age of the project and quality of source code. OpenBRR proposes eight key considerations: licensing, standard compliance, referencable adopters, availability of support, implementation language(s), third-party reviews, books, and review by industry analysts; a tool was developed to help developers follow the process \cite{deprez2008comparing,wasserman2017osspal}. Li et al. also identify eight key evaluation factors in considering open source software applications: community and adoption, development process, economic, functionality, license, operational software characteristics, quality, and support and service \cite{li2022exploring}. Cruz, Wieland and Ziegler proposed a process which considers the scenario for adopting (the user's situation), the requirements, the interpretation of collected information, and an investigation of the criteria \cite{cruz2006evaluation}. For each scenario, the authors identify the functional, technical, organisational, economical, and political requirements associated with it; for instance, when using open source software to reduce costs, economic considerations include sustainability, protection of investment, cost reduction, and division of development costs.

A literature review on software library and package selection studies found a need to develop a framework for software evaluation and selection \cite{jadhav2009review}, and there have been numerous attempts to provide automatic support for technology evaluation.
In a study of industry requirements for governance of open source libraries, Harutyunyan, Bauer and Riehle identified that such a tool should aid in searching for libraries, selecting the best library, and estimating the cost of using the library \cite{harutyunyan:2018:understanding}. 
Existing tools, such as DiffTech, LibComp, Opiner, and POME, have focused on these first two requirements \cite{huang2018tell, wang2020difftech, wang2021difftech,de2018library, de2018empirical, el2020libcomp,uddin2019automatic, uddin2022empirical,lin2019pattern}. 
%These tools rely on a variety of data sources. 
Another approach involved displaying annotated code examples \cite{yan2022concept}. What these tools have in common is that they each focus on a subset of aspects involved in the decision, and are unable to consider developer's specific priorities. Several library-specific studies have also focused on recommendations for particular libraries \cite{moataz2020android-reco,he2021migration}, but these are naturally limited to specific technologies.



% Library selection
%\subsubsection{Automation of Technology Evaluation}\label{sec:lit:automation}

%As the majority of programmers make use of tools such as search engines, code-specific search engines, and other resources in their selection of libraries \cite{umarji2008archetypal}, it is unsurprising that there have been several attempts to aid developer selection by developing tools to automate the comparison of libraries. In a study of industry requirements for governance of open source software libraries, Harutyunyan, Bauer and Riehle identified that such a tool should aid in searching for libraries, selecting the best library, and estimating the cost of using the library \cite{harutyunyan:2018:understanding}. Existing tools have focused largely on the first two requirements. DiffTech presents a side-by-side comparison of 2,410 pairs of technologies based on mining and clustering opinions from Stack Overflow \cite{huang2018tell, wang2020difftech, wang2021difftech}. The results highlight the factors that developers consider important in the comparison of two alternatives, but the approach is specific to the technologies studied and cannot be generalized to provide advice on decisions about other technologies. LibComp uses nine metrics to suggest Java libraries to developers through an IDE plugin \cite{de2018library, de2018empirical, el2020libcomp}. These metrics are automatically derived from the library repository and include measures such as popularity, release frequency, and issue response time. These may help developers answer questions about the sustainability of the software, one of the factors Spinellis advised examining in an experience paper on open source software selection \cite{spinellis2019select}, but it does not address other possible considerations, such as the appropriateness of the technology to the task. One attempt at addressing this problem involved displaying annotated code examples of user interfaces presented side by side \cite{yan2022concept}. Opiner is similar to LibComp in that it provides users with rankings based on factors, but the factors were determined through a study of users \cite{uddin2019automatic, uddin2022empirical}. Opinion mining and sentiment analysis techniques are used to create the evaluation. Meanwhile, Lin et al. also proposed pattern based mining technique `POME' with superior performance \cite{lin2019pattern}.   

%Moving from factors primarily of interest to developers to evaluation processes aimed at organizations, Wasserman et al. developed a tool to automate part of the OpenBRR process by collecting quantitative data from a source code repository, and qualitative reviews submitted to an online portal \cite{wasserman2017osspal}. Of the 74 sub-factors Li et al. identified as related to the evaluation of open source software applications, based on 170 metrics, 40 sub-factors can be regularly obtained through source code repositories, leading the authors to conclude that information provided about projects is often insufficient for consideration \cite{li2022exploring}.

%\subsubsection{Technology Specific Library Selection} Several library specific studies have been conducted focusing on specific technology or scenarios. For example, there are library recommendation techniques for Android technology \cite{moataz2020android-reco} and for library migration scenarios \cite{he2021migration}. Since Android operating system evolves faster, library upgrade is an important issue and few studies attempted to analyze the upgrade scenarios and found that 98\% libraries with security vulnerabilities are not updated, 28\% of Android applications lag in average 16 months behind latest library version \cite{aerr2017android-update, mcDonnell2013android-update-lagging, polese2022android-integration-history}. Few research have been conducted to identify the characteristics of popular library in different programming languages such as Java, Android, and JavaScript and found that popular libraries are more well-documented and can be more unstable \cite{sujahid2023popular-javascript, lima2020popular-characteristics}. A study on data science libraries revealed that developers in data science and general software engineering have different priority of library specific factors \cite{nadi2023datascience}. Literature review on software library or package selection studies has remarked that there is a need to develop a framework for software evaluation and selection  methodology \cite{jadhav2009review}.


\section{Methodology}\label{sec:methodology}
Given the absence of a library adoption model in the literature, we decided to adopt Straussian grounded theory \cite{corbin2014gt}.  
% All of the authors of this paper have at least ten years of industry experience. As such, it was also important that our formulation of the library adoption model is derived from the responses of study participants, while the model is not biased by our prior own experience. 
% Unlike classical grounded theory, Straussian grounded theory acknowledges the presence of research bias, assumptions, and motivations, and provides structured tools (constant comparison and memoing) to handle the intrusion \cite{stol2016grounded,corbin2014gt}. 
We chose to conduct semi-structured interviews of industry developers to derive our adoption model, because such interviews allow the researcher to elicit unexpected information and to evolve our questions as the study progresses \cite{HoveAnda,Seaman}. 



% Because the study involved human subjects, we obtained approval from the Ethical Review Board of our institution before conducting interviews. In total, we conducted \numInterviews interviews, interleaving data collection and analysis following grounded theory design principles. Figure~1 in the Appendix shows the overall research process.




% \begin{table*}[]
%     \centering
%     \begin{tabular}{p{.4cm}r{.4cm}l{2.5cm}l{2cm}l{2cm}l{2.2cm}r{2cm}r{1.5cm}}%{lrllllrr}
% \toprule
% P\# & Yrs & Role & Primary Tech & Continent & Industry & Company Size & Tech Size \\
%     \midrule
% P1 & 12 & Architect & Java & Europe & Automotive & 4000 & 500 &  \\ 
% P2 & 6 & Software Engineer & Python & North America & Cloud Service & 150000 & 80000 &  \\ 
% P3 & 12 & Software Engineer & Android+iOS & North America & Automotive & 30000 & 600 &  \\ 
% P4 & 20 & CEO & .NET, AWS & Asia & Broadcast Media & 60 & 54 &  \\ 
% P5 & 16 & Engr. Manager & .NET, AWS & Australia & Financial & 40 & 12 &  \\ 
% P6 & 17 & Software Engineer & Perl & Europe & Tech & 200 & 20 &  \\ 
% P7 & 9 & CTO & Javascript & North America & Data Analytics & 21 & 6 &  \\ 
% P8 & 9 & Software Engineer & Any Tech & North America & Cloud Service & 20000 & 10000 &  \\ 
% P9 & 13 & Architect & Python & North America & Web & 250 & 100 &  \\ 
% P10 & 15 & Software Engineer & Javascript & Europe & Energy & 11000 & 300 &  \\ 
% P11 & 7 & ML Engineer & Python & North America & Data Analytics & 100 & 30 &  \\ 
% P12 & 22 & Consultant & Perl & Asia & Tech & 30000 & 1000 &  \\ 
% P13 & 15 & Architect & Java & North America & Retail & 1000000 & 200000 &  \\ 
% P14 & 6 & Software Engineer & Android+iOS & Asia & Financial & 150 & 100 &  \\ 
% P15 & 22 & CTO & .Net & Asia & Enterprise & 350 & 300 &  \\ 
% P16 & 9 & Software Engineer & Java & Asia & Cyber Security & 400 & 300 &  \\ 
% P17 & 15 & CTO & Ruby on Rails & Europe & Custom Software & 35 & 6 &  \\ 
% P18 & 27 & CEO & C++ & North America & Financial & 150 & 40 &  \\ 
% P19 & 15 & Engr. Manager & Ruby on Rails & North America & Cloud Service & 150000 & 75000 &  \\ 
% P20 & 10 & Software Engineer & Android+iOS & North America & Food Service & 70 & 10 &  \\ 
% P21 & 13 & Software Engineer & Ruby on Rails & North America & CI/CD & 1800 & 900 &  \\ 
% P22 & 30 & Architect & Java & North America & Operating Sys. & 19000 & 9000 &  \\ 
% P23 & 7 & ML Engineer & Python & South America & Custom Software & 900 & 750 &  \\ 
% P24 & 6 & ML Engineer & Python & North America & Medical & 150 & 80 &  \\ 

%     \bottomrule
%     \end{tabular}
%     \caption{Interview participant's Professional profile along with the industry and size of their companies}
%     \label{tab:participants}
% \end{table*}

\begin{table}[]
    \centering
        \caption{Interview participants by years in industry, role (Arc-Architect, EM-Engineering Manager, Cons-Consultant, SDE-Software Development Engineer, MLE-Machine Learning Engineer, CTO-Chief Technology Officer), primary technology (JV-Java, PY-Python, A/IOS-Android/iOS, PE-Perl, JS-JavaScript, RoR-Ruby on Rails, Any-not limited by technology) geographic location (AS-Asia, AU-Australia, EU-Europe, NA-North America, SA-South America), and tech. people size.}
    %\begin{tabular}{p{.2cm}rp{.4cm}p{.5cm}p{.8cm}p{.5cm}p{2.2cm}rp{.5cm}}%{lrp{.5cm}p{.8cm}p{.5cm}p{2.2cm}rp{.5cm}}%Dr. Uddin
    \begin{tabular}{lrlll|lr}
\toprule
P\# & Yrs & Role & Tech & GEO & Industry & Size \\
\midrule
P01 & 12 & Arc & JV & EU & Automotive & 500  \\ 
P02 & 6 & SDE & PY & NA & Cloud Service & 80,000  \\ 
P03 & 12 & SDE & A/IOS & NA & Automotive & 600  \\ 
P04 & 20 & CEO & .NET & AS & Broadcast Media & 54  \\ 
P05 & 16 & EM & .NET & AU & Financial & 12 \\ 
P06 & 17 & SDE & PE & EU & Tech & 20 \\ 
P07 & 9 & CTO & JS & NA & Data Analytics & 6 \\ 
P08 & 9 & EM & Any & NA & Cloud Service & 10,000 \\ 
P09 & 13 & Arc & PY & NA & Web & 100 \\ 
P10 & 15 & EM & JS & EU & Energy & 300 \\ 
P11 & 7 & MLE & PY & NA & Data Analytics & 30 \\ 
P12 & 22 & Cons & PE & AS & Tech & 1,000 \\ 
P13 & 15 & Arc & JV & NA & Retail & 200,000 \\ 
P14 & 6 & SDE & A/IOS & AS & Financial & 100 \\ 
P15 & 22 & CTO & .NET & AS & Enterprise & 300 \\ 
P16 & 9 & SDE & JV & AS & Cyber Security & 300 \\ 
P17 & 15 & CTO & RoR & EU & Custom Software & 6 \\ 
P18 & 27 & CEO & C++ & NA & Financial & 40 \\ 
P19 & 15 & EM & RoR & NA & Cloud Service & 75,000 \\ 
P20 & 10 & SDE & A/IOS & NA & Food Service & 10 \\ 
P21 & 13 & SDE & RoR & NA & CI/CD & 900  \\ 
P22 & 30 & Arc & JV & NA & Operating Sys. & 9,000 \\ 
P23 & 7 & MLE & PY & SA & Custom Software & 750 \\ 
P24 & 6 & MLE & PY & NA & Medical & 80 \\ 

    \bottomrule
    \end{tabular}

    \label{tab:interviewee-profile}
\end{table}

\subsection{Participant Recruitment Strategy}
We conducted \numInterviews interviews between June 2022 and January 2023. Table~\ref{tab:interviewee-profile} provides an overview of participants.
The average number years of professional experience of the participants was 14 years. Their roles spread across the spectrum of engineering and leadership roles and in nine different countries from five continents. The tech stack also had wide varieties (e.g., C/C++, Java, Python, Android, iOS, .NET, machine learning etc.). Many of our participants were working in large corporations (e.g., Google, Microsoft, etc.) and open source projects. Among all the interviewees, three experts identified themselves as female. Our participants covered 16 application domains including specialized regulated areas as such health, finance, cybersecurity, and broadcast media. 

 % An important part of grounded theory is theoretical sampling of the concepts. It means that after every interview, we had to perform data analysis of the interview, identify the concepts which needed further elaboration, and select the next few interviewees to improve the probability of getting answers regarding emerging concepts. Before conducting interviews, we performed initial screening of the participants through a short phone call (around 5 to 10 minutes). After we identified that a participant could provide relevant insights, we sent them a formal interview invitation. 
 % In this way, we reached out to 38 potential participants for initial screening of emerging concepts through the professional network of the authors. 



\subsubsection{Theoretical Sampling for Recruitment}
We started with an architect (participant P1) from our professional acquaintance who had twelve years of experience including designing and developing from scratch a payment system in a 260 human-year project spanning over six years. We knew that they had to select a huge number of libraries throughout this development. Our first interview lasted  115 minutes, as we were developing initial concepts around all of our research questions, while subsequent interviews averaged 64 minutes. % range 42-115
After analyzing the initial interview data, we realized that we needed more information about commercial factors (particularly licensing) and integration process of a library. Hence, we chose our next participant (P2) who was working in large organization (\#engineers $\geq$ 80K), and who had experience on licensing concerns about software libraries. In this way we continued the theoretical sampling to saturate all the relevant concepts. The motivation for selecting every participant is provided in the Appendix. 
% In most of the cases, we were able to extract insights about target concept from the interviewees. For example, we wanted to dig deep into the library maintenance issues and we identified mobile applications need continuous update because of the yearly operating system update. Hence we reached out to a mobile developer P14 who actively manages applications in all mobile OS platforms and in the interview, they provided us significant insights about maintenance and migration issues. 


\subsubsection{Concept Saturation over the Interviews}
%\begin{figure*}
    \centering
    \includegraphics[scale=0.9]{images/saturation.pdf}
    \caption{Heatmap of concept saturation over the interviews. Red refers to the higher discussion of a concept by an interviewee, the number refers to how many times the interviewee has discussed a concept. Green refers to the lower discussion of concepts by an interviewee. Thick black-bordered cells refer to the most discussion reference of a concept among all participants. For example, the review concept under the selection process category was discussed 26 times in P08's interview and hence the corresponding cell is marked by the thick border. The last two rows show the number of concept saturation by each participant and the cumulative number of concepts saturated till that participant. For example, the interview with P08 alone saturated 3 concepts (denoted by the thick boxes in the P08 column) and after the P08 interview, total of 17 concepts were saturated.}
    \label{fig:saturation}
\end{figure*}
Our work was initiated with the question RQ1: ``What are the major steps developers follow during their adoption of a software library?'' We soon realized that the process is influenced by certain conditions and factors, leading us to ask RQ2: ``What conditions and factors influence the library adoption process?'' 
%As we continued to conduct interviews, our concepts started to saturate. Though many interviewees contributed towards the saturation, a few interviewees contribute most, and after such an interview a concept saturates very quickly. 
By time we interviewed P8, we also realized that there were specific library selection principles and many pros and cons of libraries depending on the scenarios. Also after hearing P9's concerns about third-party libraries, we started to discuss about benefits and barriers more. This led to RQ3: ``What guiding principles do developers follow in library adoption?'' These concepts were well understood by the time we interviewed P22. At the end of the interview of P24, we did not get any new dimensions or properties of existing concepts as they were all well saturated.

%\textcolor{blue}{[Minaoar]}
%Our work was initiated by the following question: \textit{\textbf{RQ1:} How developers adopt third-party libraries?} After understanding the adoption process, we realized that the process is influenced by certain conditions and technical factors. Then we tried to answer \textit{\textbf{RQ2}: What conditions and technical factors influence the library adoption process?} Finally, we were curious why developers change adoption process based on the conditions and factors? Is there any guiding principles developers follow for library adoption? Then our final question was \textit{\textbf{RQ3:} What guiding principles developers follow in library adoption?} \ab{I'd put this in the methodology where you're talking about GT}

% \subsubsection{Repeated Interviews for Open Questions}
% However, in few cases, we could not find elaboration of our target concept from the interviewee and had to plan for other participants who could answer the questions we had. For example, we expected that P12, who specializes in DevOps process establishment in tech companies, could share experience on continuous monitoring of libraries for security concerns. However, P12 explained that in most of the cases when they consulted the organizations, they could never reach up to that matured security monitoring level, their target was to achieve continuous functional test coverage up to 50\%. Though we could not elicit the target concept, we could collect valuable insights about the reservations organizations have towards open source libraries. Then we interviewed P16, who was a certified security professional actively developing security products, to know their security protection against software libraries. They again did not follow any continuous monitoring process, they only ran security audits once a year. Finally, we got a clear conception on continuous security monitoring for third-party libraries from P17. 

% \subsubsection{Profile of the Interviewees}
% Following the theoretical sampling, we stopped the interviews when we had properties and dimensions of all concepts. The average number years of professional experience of the participants was 14 years, and their roles spread across the spectrum of engineering and leadership roles, including engineer, architect, consultant, CTO and CEO who are working in nine different countries on five continents. The tech stack also had wide varieties, e.g., open source library intensive languages such as Java, Python, Perl, JavaScript, Android, and iOS, and also historically less library focused languages such as C/C++ and .NET. Many of our participants have professional experience of working in large corporations such as Microsoft, Google, Amazon, AWS, Oracle, Cisco, IBM, and open source projects such as GitLab, RedHat Enterprise Linux, and Azure Communications Services, as well as leading startup and small size companies with 6 to 10 developers. Three of the participants described more than 90\% of their full time work in open source projects. Among all the interviewees, three experts identified themselves as female. Our participants also covered a wide range of 16 application domains including specialized regulated areas as such health, finance, cybersecurity, and broadcast media. 

% \subsubsection{Interviews}
% We conducted \numInterviews interviews between June 2022 and January 2023. All interviews were conducted online using Microsoft Teams for easy transcription. All interviews were both audio and video recorded except one. Interviews lasted from 42 minutes to 115 minutes with an average of 64 minutes per interview. Because of lack of accuracy of auto-transcription, each interview was manually corrected. % In average, it would take around 12 hours to conduct and analyze each interview (interview process 2 hours, transcription 4 hours, coding 4 hours, memoing 2 hours) amounting to around \td{288} hours of manual work.

\subsection{Interview Data Collection and Analysis}
All interviews were conducted online using Microsoft Teams for easy transcription. All interviews were both audio and video recorded except one. Interviews lasted from 42 minutes to 115 minutes with an average of 64 minutes per interview. Because of lack of accuracy of auto-transcription, each interview was manually corrected.  
Qualitative data analysis (QDA) was interwoven with interviews. We used the qualitative data analysis tool MaxQDA
\cite{website:maxqda} for coding, memoing and diagramming. The coding process consisted of open coding, axial coding, and theoretical integration. 
% Coding and categorization was done through the hierarchical Code System feature, while interview segments were annotated with In-Document Memos. The Code Matrix Browser was used to track emerging concepts. We used the In-Document Memo feature to sort memos and compare and refine the concepts to generate central categories. MaxMaps was used to create a hierarchy of codes and the process flow of library selection.

\subsubsection{Open Coding and Memoing} During each interview, we continuously took field notes so that we could identify our concerning points. After each interview, we made summary memos with the new concepts that emerged, the properties of existing concepts that saturated, and the questions that arose. After first couple of interviews, the concepts had been identified and we were able to begin open coding after each interview. Initially, all codes were categorized under either of the process, factors, sources, and conditions concepts which directly came from our research questions. As the number of interviews increased, we also increased memoing for concepts and questions which were emerging. The memos contained our thoughts regarding the opinions shared by the interviews. We were also constantly comparing opinions provided by a new participant with previously analyzed data. 

\subsubsection{Axial Coding} After coding and analyzing the first four interviews, we started to discover the intersection points of different concepts using axial coding. For example, we identified that some selection factors are related with technical issues such as \code{ease of use}, \code{performance}, and \code{compatibility}, whereas a few factors are not dependent on the software of the library itself, such as \code{active development}, \code{community support}, paid \code{customer support}, which mostly relate with central concept of \code{support and maintenance}. Similarly, \code{company culture} and \code{company technology} had a central theme of \code{organizational influence}, which differs from developer's \code{personal background}, \code{experience level} or \code{personal motivation} which fall under the \code{individual influence} category. Using axial coding analysis, we created another layer of categories to group similar factors and conditions. 

% \subsubsection{Memoing} As the number of interviews increased, we also increased memoing for concepts and questions which were emerging. The memos contained our thoughts regarding the opinions shared by the interviews. We were also constantly comparing opinions provided by a new participant with previously analyzed data. Sometimes if there is any contradiction among interviewees, we wrote memos how the experts were contradicting and what were the underlying reason they expressed different opinions. For example, P10 (a web developer) and P14 (mobile app developer) shared differing experience regarding library version upgrades, and we noted this memo: \textit{"According to P10's experience, if a major library upgrades its version, then other dependent libraries also keep track of the upgradation and keep their (dependent) libraries updated as well. However, from P14's experience of mobile OS upgradation, they faced deprecated libraries which were no longer supported. Why they are experiencing opposite situations? Is it because mobile operating systems upgrade more frequently than backend infrastructure and mobile library developers cannot keep up with the changes and leave the projects?"}

\subsubsection{Theoretical Integration} 
% The version of Straussian grounded theory we employed, published in 2014, differed from the earlier version by placing emphasis on theoretical integration rather than selective coding to generate the core concept \cite{strauss1998basics,corbin2014gt}.
Diagrams and memos helped us conduct theoretical integration to generate the core concept of our research. For example, we started the interviews and analysis to explore how the developers adopt libraries. As the analysis progressed, we attempted to generate a core concept by sorting the memos and drawing interactive diagrams. We realized that unless we deeply understood why developers use libraries, we could not generate a core concept. After we identified developers' motivations and concerns for third-party libraries, the core concept emerged as the guiding principles of software library adoption which in an organization can guide a developer to make a decision by employing the library adoption steps and by considering the factors and conditions that influence the steps. 
%\gias{give a concrete example of how a guiding principle is derived using theoretical integration} 
For example, few developers wanted to talk to people when they were under tight deadline \emph{``the way of choosing libraries was actually talking to peers because we were in a rush to deploy" (P16)}. So we thought \code{Meet deadline} is a core concept. However, we also found that even when there was no deadline, some developers still reached out to their peers, because they did not want to spend time on searching library: \emph{``They [friends or colleagues] already did it, right? They can just tell you do this." (P11)}%{P11}
They wanted to \code{make their life easy}. Connecting these two motivations, we came to conclusion that both of them followed a common principle of \code{Just Do It}. 
% but in completely different conditions. In this way we could explain all types of developer decisions by following the guiding principles and these became the core concept of our library adoption model.

% , which was related with other concepts of the library factors, selection process, and conditional influence on the adoption.

% When we were asking why developers use libraries, suddenly an interviewee (P12) responded, \textbf{'Let me tell you why developers DO NOT use libraries'}. That was a ureka moment for us that led to the core concept of double-edged sword, and explained why developers consider so many factors and their complex organizational and technical conditions. Later using the sorted memos and interactive diagrams, we performed the theoretical integration to generate the theoretical framework of library adoption based on the core concept of a double-edged sword - \textbf{'win the battle with a double-edged sword, be careful lest it cuts you back'} or 'make your life easy, use a library instead, just be careful of the baggage it brings in.'

%\subsection{Data Availability}
%We include the following documents in our supplementary material: \todo{add reference to supplementary material - final version on zenodo \cite{website:replication-package}}
%\begin{itemize}
%\item Interview script
%\item Invitation emails containing adjusted interview script
%\item Consent form
%\item REB approval
%\item Final codebook exported from MaxQDA in Excel and MaxQDA format
%\item Grounded theory evaluation results of the study
%\item Appendix
%\end{itemize}
%In order to preserve the anonymity of our participants and to adhere to our ethics approval, we do not share a list of interview participants or the interview transcripts.


\input{D.Framework.tex}

%\section{Guiding Principles and Process of Library Adoption}
%\label{sec:taxonomy}
%\begin{figure*}
%    \centering
%    \includegraphics[scale=0.85]{images/process.pdf}
%    \caption{Concepts related with library adoption process and relevant information sources}
%    \label{fig:process}
%\end{figure*}
% Using open coding and constant comparison of \td{24} interviews, we discovered total 83 library adoption concepts. By conducting axial coding, we categorized these concepts into 19 categories under 4 major categories of adoption steps, surrounding conditions, library specific factors, and information sources. Figure \ref{fig:taxonomy} shows the hierarchy of all 83 library adoption concepts. Numbers shown beside the major categories in Figure \ref{fig:taxonomy} refer to the total number of concepts under that category (e.g., 18 concepts under 'steps' category).



\begin{figure*}
    \centering
    \includegraphics[scale=0.78]{images/process.pdf}
    \caption{Concepts related library adoption process and information sources}
    \label{fig:process}
\end{figure*}

\subsection{Library Adoption Process (P)}\label{sec:phases} 
The adoption of a third-party library consists of five steps, as depicted in Figure~\ref{fig:process}. The steps are: search for library information, compare available libraries, review libraries along with other teammates or teams, integrate the library into the application, and maintain the library through the process of the host application. There were 18 concepts associated with \code{steps of library adoption process} in our code system. Figure \ref{fig:process} shows the relationship between relevant codes and categories for the library selection process and information sources consulted during the process.
%Figure~3 in the Appendix shows the relationship between relevant codes and categories.

\subsubsection{Step 1. Search} \code{Search} entails \code{identifying the problem}, \code{talking to people}, and \code{performing an online search}. 
\subsubsection{Step 2. Compare}
Next, developers \code{compare} available libraries by \code{comparing alternatives}, \code{exploring candidates}, and \code{selecting the outstanding library}. The comparison concept is explained by the interview participant, P15:

\qi{You go to Stack Overflow and you find some articles there that refer to some of the libraries. From there, you go into those libraries spaces and GitHub and then you take a look at it.}{P15} 

% \gias{let's add something from Fig 6}

\subsubsection{Step 3. Review} \code{Review} of the library can involve multiple concepts: \code{team discussion}, \code{design/code review}, \code{consent process}, \code{convince developers}, \code{convince non-developers}, and \code{stakeholder consultation}. Depending on the company size and practices, developers may request approval from dedicated teams who review the security and license issues of third-party libraries: 
\qi{So we have the DevOps team\ldots they have a way to check the security also any vulnerability issue altogether and also if we are actually using it in the proper license.}{P11} 


\subsubsection{Step 4. Integrate} To \code{integrate} a library can also be a gradual process, consisting of \code{proof of concept development}, \code{integration}, and \code{gradual adoption}, as illustrated here:
\qi{[We] do some proof of concept for some libraries. Then for the adoption phase we normally take it slowly, like for example, when we want to introduce Hikary CP instead of Tomcat, we have to apply this library to one or two services, then gradually move to the wide adoption of that library.}{P01} 

\subsubsection{Step 5. Maintain} Finally, to \code{maintain}, developers may \code{scale up usage}, perform \code{post integration maintenance}, and \code{migration}. Once the library has been integrated, developers may need to keep up with the latest versions:

\qi{There is no limit on improvement\ldots You can opt-in for new 
versions or you can ignore the new versions. Usually, major versions might have something which can break your code. But you don’t upgrade without any reason.}{P05}
%\todo{@Minaoar - I took this quote from elsewhere, please replace it if it isn't coded under maintain} \minaoar{looks good}.


\subsection{Information Sources (S)} \label{sec:sources}
Developers seek out information about libraries from five categories of information sources when making an evaluation: \code{human sources}, \code{online search and articles}, \code{question answer sites}, \code{repositories}, and \code{organizational sources} internal to the company. Within these five categories, we identified 15 concepts, whose hierarchical relationship is shown in Figure \ref{fig:process}.

% Of the \code{human sources} available to developers, there are \code{known people}, the developer's \code{regular study}, and \code{developer meetup}. Which sources the developer favors can vary, as illustrated by these two quotes:
% \qi{The first thing is not opinion from other people, rather from developer's daily study. Developers always look for [learning] things, right?}{P01}
% \qi{The very first step before going to searching is I'm reaching out to my colleagues.}{P11}

For \code{online search and articles}, developers can turn to \code{web search}, \code{online article}, \code{influential blog}, and \code{case studies}. Alternately, they may turn to \code{question and answer sites} such as \code{Stack Overflow} or \code{other Q\&A sites}:
\qi{Stack Overflow is a huge resource for seeing what different people recommend\ldots and also seen a lot on things like Quora and Reddit where you say what's the best library for doing X and people will list out a couple of different options there.}{P07}

Another source of information is \code{repositories}, which can consist of the \code{direct library vendor}, \code{package repository}, or \code{source repository and official website}. Finally, developers can turn to internal \code{organizational sources}, namely \code{existing source code} and \code{organizational content}. 
\qi{[We have an] internal GitHub. Then there are internal search engines, also there are some question answers like Stack Overflow\code I think that is true for all these big corporations.}{P02}

One participant shared information about a very unique data source ChatGPT\footnote{\url{https://openai.com/blog/chatgpt}}, a chatbot supported by large-language model:
\qi{One more vector of discovery that I have been lurking with the last few weeks is I'm giving a prompt to ChatGPT or my team is giving a prompt to the copilot, not Googling first, but rather prompting it first and seeing what it gives and then copying and pasting that code into Google and finding out actually what these things actually do and actually are they valid or not. And surprisingly the results of the prompts give you fairly accurate suggestions for the libraries or the APIs that you need to use. I think it's yes I would say probably 70\% of the time it's useful. 30\% of the time they actually give you false names of APIs or functions that actually do not reside in reality or those packages do not exist. }{P16}

\subsection{Conditions Influencing Library Adoption (C)}
\label{sec:conditions}

\begin{figure*}
    \centering
    \includegraphics[scale=0.75]{images/conditions.pdf}
    \caption{Factors and conditions influencing adoption process (gray boxes are concepts that are found for the first time in our study)}
    \label{fig:conditions}
\end{figure*} \fig\ref{fig:conditions} summarizes the concepts related to external (environmental) and internal conditions that can influence the library adoption process. 
% Our second research question was ``What conditions and technical factors influence the library adoption process?'' In order to answer this question, we turn to the conditions of library adoption and the factors of library selection criteria.

The library adoption process can be influenced by environmental, organizational, team-specific, individual, and technical scenario-specific conditions and opinions. These conditions affect which guiding principles are most appropriate. In the major category of \code{conditions}, we identified five categories: \code{environmental}, \code{organizational}, \code{team influence}, \code{individual}, and \code{technical}, which are depicted in relation to the process of adoption in Figure~\ref{fig:framework}. There were 23 concepts associated with these categories, as shown in \fig\ref{fig:conditions}.

\code{Environmental} conditions include the \code{geographic} and \code{legal} landscapes. For example, regulatory requirements influence the legal conditions under which the company operates:
\qi{In Germany you have to report a security breach in your company\ldots you have to pay two percent of the revenue if a security breach happens and your data gets leaked.}{P17} 
Geography can influence selection, outside of legal requirements:
\qi{{\ldots} everywhere it's not still functional programming, it's not widely adopted. I recently migrated from Asia to Europe and I never saw this trend widely adopted in our previous companies. But here [in Europe], I've seen a lot of people are very interested in that functional programming paradigms. So, to choose the libraries, maybe their background, their geography, their location, all of these things can have some impact.}{P01}

Organizations can also enforce policies for library selection. Under \code{organizational} conditions, we observed \code{investment capacity}, \code{company culture}, existing \code{company tech}, and \code{application domain}. There can also be a \code{team influence}, based on \code{team capability}, \code{expert/lead opinion}, \code{operations opinion}, \code{peer opinion}, \code{stakeholder opinion}, and \code{product/project manager opinion}. Team capability is demonstrated with the following quote:
\qi{I went for Vue because most of the developers in my company were mostly back-end developers and I found Vue is very back-end developer friendly.}{P04}

\code{Individual} developers can also have personal characteristics which are independent of technology but which influence library selection. These can include \code{personal background}, \code{personal motivation}, and \code{experience level}. An example of personal motivation is:
 \qi{The excitement of trying some new library was also fairly motivating to keep my skills up to date\ldots}{P06} 

 Finally, there can be \code{technical} conditions, namely a severe issue in production (\code{stuck in production}), having a \code{tight deadline}, \code{migration to a different library}, \code{criticality of feature}, \code{new feature}, \code{tech stack upgrade}, \code{library upgrade}, and \code{OS upgrade}. The influence of criticality can be seen here:
\qi{If the feature is too business critical, then it goes through an even more rigorous decision-making process than the other where, for example, you are just trying to choose a library to show an image and crop it. So that is not so business critical.}{P15}


\subsection{Library Selection Factors (F)}
\label{sec:factors}
Based on the selection process and conditions, developers choose the selection factors for a library, as depicted in Figure~\ref{fig:framework}. We identified four categories of \code{factors}: \code{software factors}, \code{commercial factors}, \code{maintenance factors}, and \code{external factors}. Under these, we found 28 concepts, whose relation to the categories is shown in  \fig\ref{fig:conditions}.

In the category of technical \code{software factors}, we see \code{compatibility}, \code{stability}, \code{flexibility}, \code{capability}, \code{security}, \code{performance}, \code{ease of use}, \code{ease of installation}, \code{size of library}, and \code{interesting interface} as concerns. Several of these considerations are provided by P15:
\qi{We have to understand how much memory it is gonna use. How much time does it take to execute? Or how much CPU is gonna use? Is this library compatible with the development environment? How is the thread safety within the library?}{P15}

After technical factors, developers may have to consider \code{commercial factors} such as \code{license}, \code{cost}, \code{dependency}, \code{roadmap}, \code{open source}, \code{documentation}, and \code{demo availability}. An example of the impact of demo availability on the selection process is:
\qi{And one thing I prefer over other GitHub repositories is that if it has a working demo available on the web. For example, if it is a front-end library, then it has a UI. If it's a back-end library, it has some kind of documentation available like Swagger so that I can make a call to the library and see how it works. If some kind of live demo is available, then I feel more confident about it.}{P15}

\code{Maintenance factors} include whether the library source code is under \code{active development}, enjoys \code{community support}, is \code{supported by a reputed organization}, has a \code{large community}, offers \code{customer support}, and is \code{supported by own organization}. 

Finally, \code{external factors} can come into play. These include \code{popularity}, \code{search engine ranking}, \code{familiarity}, \code{used by reputed organizations}, and \code{detailed benchmark}. The following quotations demonstrate how popularity and familiarity affect selection:
\qi{We cannot compare sixty million with one thousand downloads. So this [frequently downloaded] library is obviously a choice.}{P04} 
 \qi{Oftentimes what happens is that the decision or the choice gets influenced by an individual's bias or familiarity or previous experience with one particular product or service}{P08} 

\subsection{Barriers of Library Adoption (B)}
Before being able to integrate a library, there are certain entry barriers that an organization or a developer may try to avoid for considering library usage at all. Such obstacles can be caused by organizational policy, and technology or even by developers' mindset, and experience. We identified a total of seven challenges deriving from three major sources of the organizational culture, individual traits, and industry conditions.
% \subsection{Recommendations}

%\subsubsection{Lack of Organizational Policy}
\nd\bf{\ul{B1. Lack of Supporting Process.}} To manage third party open source libraries, many companies \code{lack supportive process} such as an open source program office. In some cases, organizations may be unclear about what their developers can use and cannot use from third parties, and instead of embracing external libraries, they can be rather rigid: \qi{I was in organizations where they were really so afraid of the legal ramifications of using open source that I had to sit through hours of filling out forms to get approval for each library and so that can be a limiting factor if you're a developer. \textbf{You usually you would spend 100 hours developing something instead of one hour trying to fill out some legal form.}}{P12} 

\nd\bf{\ul{B2. Infrastructural or Technological Limitations}} can also cause difficulty in adopting new libraries. Some organizations \textit{``try to limit the access to the Internet.''}$_{P12}$ and then it's not that easy to install third-party libraries from repositories such as NPM used in Node JS. Also, in some cases, because of the current tech stack of the application, integrating a new library can cause a lot of difficulties. A participant who developed a C++ application in the  Windows environment for over 27 years shared their experience: \qi{C++ libraries are notorious for the amount of work to integrate them, to get them to compile on. Like if the library was designed for Linux, it's really hard to get them actually to compile with Visual Studio, to get your compile flags all right, to get all your dependencies right.}{P18} 

\nd\bf{\ul{B3. Inclusivity Barrier.}} Though library selection is a team decision, sometimes \code{lack of inclusivity} hinders the consideration of all opinions.
Developers seek information from a variety of sources. If the organization does not have a very welcoming, inclusive culture, critical analysis of libraries can be ultimately influenced and driven by outspoken people. 
\qi{Let's say some developers are more vocal, some people are better in writing, and some people are comfortable reading a material before joining a meeting. Moreover, in some cultures, people understand more from the context, and in some other cultures, people understand only from the word that has been spoken. 
In an ideal scenario, every team member can contribute using their preferred communication style - vocal or non-vocal, writing or reading. But in real life, we are a bit diverse from that ideal world where we can make it really inclusive. Oftentimes what happens is that the decision or the choice gets influenced by developers' bias who are more vocal.}{P08}

 \nd\bf{\ul{B4. Lack of a Learning Culture}} across the development team also creates an unbalanced team discussion in situations like library selection. Even when the culture promotes openness, development teams often have few enthusiastic developers who love to explore and whose opinions might have a disproportionate weight in discussions. 
 \qi{there are various types of engineers, some people are very interested in technology careers. So they're always keen to update themselves on the latest technology. Some developers are like regular. They're not too much serious, they're moderate, but they don't go the extra mile. So you have to identify those types of personalities in your team and try to encourage them to get training materials, and read books so that they can also contribute to team discussions.}{P05}


\nd\bf{\ul{B5. Lack of Experience.}} \code{Developers' lack of experience} in software development can allure them to solve some problems by themselves for which there are already robust libraries. They cannot estimate how much work would be needed to implement by themselves: \qi{Then there is the ‘not invented here’ syndrome. So many people think that they can do it better... So you say - oh, I can write just some code doing this, and investing the time to learn the third party library seems like too much investment... And later on, it will turn out that your small project will grow. And then you will basically reimplement whatever it's outside. But at the beginning, it seems that it's easier to do this [implementation] than learning [new library].}{P12}


\nd\bf{\ul{B6. Change-Averse Mindset.}} The analysis also reveals that there can be a \code{mindset of developers} that can hinder them from learning new technologies or libraries and stick to their own development: \qi{So I saw people who were happy to learn and start trying new things and introduce new things [libraries]. And some who were really, really refusing things. And my description is that it's sort of the internal age of the person, so not the external, not your real age, but sort of like it's a mentality thing.}{P12}

\nd\bf{\ul{B7. Lack of Comparison Tools.}} \code{Lack of availability of comparison tools} for libraries make developers confused to choose a library after searching online. With the organization's support, policy, and team's willingness to collaborate and explore libraries, there are always challenges of finding out the appropriate library by going through numerous articles, documents, and reviews online. There still is a lack of guiding tools that can support library selection (or in general, technology selection) by summarizing a large amount of data according to the team's priorities. Developers cannot rely on the existing summarization research outcomes since the detailed reference and quality assurance of those tools still are not considered worthy of industrial usage: \qi{How do people decide? There are like 50 things implementing the same thing. Which one should I use? Then there is analysis paralysis when you have an abundance of choices and then you can't make a choice... I don't see any tools really for that. Maybe there are. I guess there should be or I don't know.}{P12}

\subsection{Decision Patterns of Software Library Adoption (D)}
\label{sec:gp}

All of the concepts of the library adoption process described in this section have complex interactions which we captured in the six \principle. All of the patterns represent an abstract solution to a recurring problem \cite{riehle:2021:pattern}. On the basis of internal and external conditions, certain advantages and disadvantages of libraries become relevant to developers, and they follow a decision pattern accordingly. Thus the \principle\space show interaction among the selection factors, information sources, and the adoption process. 
 \qi{It's not that always you have to choose a library or framework which is technically the best. Rather capabilities of the library and the organization, the domain, the people, the timing, everything influences that decision.}{P01} 

It must be noted that conditions do not come in isolation, and \principle\space also do not activate exclusively. Rather, in practical scenarios, developers are always facing multiple, even conflicting situations, and are balancing between \principle\space to make their final decisions.

% Due to space limitations, we are only able to present some of the guiding principles in depth. All six can be found in the Appendix.

%%%%%%% GP 1
\nd\bf{\ul{D1: Just Do It.}} When developers are concerned primarily with time-to-market or are not that interested in long-term maintenance, they may opt to make use of third-party libraries which are \qqw{F}{easy to use} and \qqw{F}{easy to install}, as illustrated by the following quotes. 

\qi{[The library] makes our life simpler. The application development definitely gets faster by using that.}{P15}
\qi{It was going to solve a particular promotion or something, and it was going to be retired. So usually the long-term maintainability was not a factor.}{P06}

\begin{practice}{1}{Just Do It}
\actor{Developers}
\condition{Faster go to market is critical for business. Developers are often rewarded for delivering minimum viable product on time.}
\concern{How the developers can meet the deadline with relatively less effort?}
\solution{Use a third-party library that reduces the work load and delivery can be done on time}
\consideration{The library should be easy to use, easy to install, popular, and even better if developer is already familiar with it.}
\steps{Find libraries, compare them, and choose one according to the consideration. Take support from known people or use search engine or Stack Overflow}
\example{For startups, a lot of it [priority] is just speed to market and how much resources is gonna eat up using any specific library.}{P07}
\end{practice}

% \begin{table}[]
%     \centering
%     \caption{Scenario of GP1: Just Do It}
%     \begin{tabular}{p{1cm}p{6.5cm}}
%          \toprule
%             \textbf{GP} & \textbf{Just Do It} \\ 
%             \midrule
%             Actor(s) & Developers \\ 
%             Context & - Faster go to market is critical for business. Developers are often rewarded for delivering minimum viable product on time.
%             - Developers are in their early career and assume higher satisfaction on faster delivery. \\ 
%             Concern & How the developers can meet the deadline in relatively less effort? \\ 
%             Solution & Use a third-party library that reduces the work load and delivery can be done on time \\ 
%             Consider-ation & The library should be easy to use, easy to install, popular, and even better if developer is already familiar with it. \\ 
%             Steps & Find libraries, compare them, and choose one according to the consideration \\ 
%             Support & Take support from known people to know about such library or use search engine, Stack Overflow \\ 
%             Example Trace in Data & For startups, a lot of it [priority] is just speed to market and how much resources is gonna eat up using any specific library. (P07) \\ 
%          \bottomrule
%     \end{tabular}
%     \label{tab:gp1-scenario}
% \end{table}
 %% Ann: I exclude this one because it is fairly evident (to save space)


%%%%%%%% GP 2
\nd\bf{\ul{D2: Reuse Robust Component.}} Developers working on long-term or complex products prefer to choose \qqw{F}{stable} which are \qqw{F}{actively maintained} for a long period and supported by a \qqw{F}{large community}. The \principle\space is \code{Reuse Robust Component}. Examples of its use are:
 
\qi{%Don't reinvent the wheel. 
I want to use as much as already developed, tested, and robust software in my solution\ldots the main thing is that reusability and having stability in the application inherently out-of-the-box by using a stable, robust library}{P22}
\qi{With a project like ours [C++ code application developed over 27 years], the main time we adopt a library is when it is an implementation of a tech spec. OpenSSL, there's a tech spec for SSL. Here's a whole library that has been carefully implemented against the spec.}{P18}

\begin{practice}{2}{Reuse Robust Component}
\actor{Developers, Senior Developers}
\condition{Mature organization with stable source code and release pipeline. Application performance and maintainability is a big concern for the team.
Developers with higher experience are careful about the quality of the application source code.}
\concern{How can the application avoid boiler plate code and follow best design principles?}
\solution{Use a trusted proven third party library that will keep the code clean and manageable.}
\consideration{The library should be open source, trusted by community, stable and provide apppropriate performance metrics required}
\steps{Find and compare libraries, review thoroughly by more than one developer. Look into the library's source code repository to analyze the stability, quality of the library, and also consider reputed technical blogs.}
\example{So [this large corporation] as a whole is actually built on the open source libraries that are suitable for our use cases. But that actually has been one of our primary focus as well. If you find a library, use it; only build if you can’t find anything.}{P13}
\end{practice}

% \begin{table}[]
%     \centering
%     \caption{Scenario of GP2: Reuse Robust Component}
%     \begin{tabular}{p{1cm}p{7.5cm}}
%          \toprule
%             \textbf{GP} & \textbf{Reuse Robust Component} \\ 
%             \midrule
%             Actor(s) & Developers, Senior Developers \\ 
%             Context & - Matured organization with stable source code and release pipeline. Application performance and maintainability is a big concern for the team.
%             - Developers with higher experience would be careful about the quality of the application source code. \\ 
%             Concern & How the application can avoid boiler plate code and follow best design principles? \\ 
%             Solution & Use a trusted proven third party library that will keep the code clean and manageable. \\ 
%             Consider-ation & The library should be open source, trusted by community, stable and provide apppropriate performance metrics required \\ 
%             Steps & Find and compare libraries, review thoroughly by more than one developer. \\ 
%             Support & Look into the library's source code repository to analyze the stability, quality of the library, and also consider reputed technical blogs. \\ 
%             Example Trace in Data & So [this large corporation] as a whole is actually built on the open source libraries that are suitable for our use cases. But that actually has been one of our primary focus as well. If you find a library, use it; only build if you can’t find anything. (P13) \\ 
%          \bottomrule
%     \end{tabular}
%     \label{tab:gp2-scenario}
% \end{table} %% Ann: I exclude this one because it is fairly evident (to save space)

%%%%%%% GP 3
% \begin{practice}{1}{Just Do It}
\actor{Developers}
\condition{Faster go to market is critical for business. Developers are often rewarded for delivering minimum viable product on time.}
\concern{How the developers can meet the deadline with relatively less effort?}
\solution{Use a third-party library that reduces the work load and delivery can be done on time}
\consideration{The library should be easy to use, easy to install, popular, and even better if developer is already familiar with it.}
\steps{Find libraries, compare them, and choose one according to the consideration. Take support from known people or use search engine or Stack Overflow}
\example{For startups, a lot of it [priority] is just speed to market and how much resources is gonna eat up using any specific library.}{P07}
\end{practice}

% \begin{table}[]
%     \centering
%     \caption{Scenario of GP1: Just Do It}
%     \begin{tabular}{p{1cm}p{6.5cm}}
%          \toprule
%             \textbf{GP} & \textbf{Just Do It} \\ 
%             \midrule
%             Actor(s) & Developers \\ 
%             Context & - Faster go to market is critical for business. Developers are often rewarded for delivering minimum viable product on time.
%             - Developers are in their early career and assume higher satisfaction on faster delivery. \\ 
%             Concern & How the developers can meet the deadline in relatively less effort? \\ 
%             Solution & Use a third-party library that reduces the work load and delivery can be done on time \\ 
%             Consider-ation & The library should be easy to use, easy to install, popular, and even better if developer is already familiar with it. \\ 
%             Steps & Find libraries, compare them, and choose one according to the consideration \\ 
%             Support & Take support from known people to know about such library or use search engine, Stack Overflow \\ 
%             Example Trace in Data & For startups, a lot of it [priority] is just speed to market and how much resources is gonna eat up using any specific library. (P07) \\ 
%          \bottomrule
%     \end{tabular}
%     \label{tab:gp1-scenario}
% \end{table}
 
\nd\bf{\ul{D3: Maximize Flexibility.}} While inspired by the principle of architectural improvement and consistency, architects adopt third-party libraries when absolutely necessary for specifications. Project P18 worked was such an example where they had less than 20 libraries in a two million line source code. Even when they \qqw{P}{integrate} libraries, they would wrap it under their own structure: 

\qi{We will create just a thin wrapper, that ends up looking like the rest of our platform. The simple source file [wrapper API] is protecting the two million lines of code from the idiosyncrasies of this one particular library.}{P18}

\qi{Just go with the smaller libraries that solve a very particular problem so that we have full control. But if we integrate with the framework then we always need to solve with that framework because we cannot go outside of that framework until we completely remove that framework}{P9}
\begin{practice}{3}{Maximize Flexibility} % Was Improve Application Structure
\actor{Developers, Architects}
\condition{Large scale software applications can have their own structure which evolved over time and may follow some software design principles. System designers want to use new libraries in a way that improves the application structure, or at least does not deteriorate it.
When implementing critical features, architects often plan for extendability and maintainability. A third party application can pose both a risk and advantage in this regard.}
\concern{How can the application architecture  be improved or be protected from unwanted rigidity by using a third partly library?}
\solution{Use a library that just fits right with the application in terms of the size and flexibility. Consider wrapping the third-party library in an API to allow replacement of the library without affecting dependent code.}
\consideration{The library should be flexible and should not be too large in size compared to the required functionality.}
\steps{Find, compare, and review libraries. In addition, conduct design review to assess impact on architecture. Review internal organizational content for design principles, and study study online articles for suggestions.}
\example{The moment you have to bring something in because there are new requirements is the time to assess how you've structured your application and does it still serve you and your customers or the business requirements you have. So that’s the opportunity to look at the structural aspect of the application and make sure you do want to avoid changing it or maybe
it's the time to change it.}{P22}
\end{practice}

% \begin{table}[]
%     \centering
%     \caption{Scenario of GP3: Improve Application Structure}
%     \begin{tabular}{p{1cm}p{7.5cm}}
%          \toprule
%          \textbf{GP} & \textbf{Improve Application Structure} \\ 
%          \midrule
%             Actor(s) & Developers, Architects \\ 
%             Context & - Large scale software applications can have their own structure which evolved over time and may follow some software design principles. System designers would be using new libraries in a way that improve the application structure, or at least does deteriorate it.
%             - When implementing critical features, architects often plan about extendability and maintainability. A third party application can pose both a risk and advantage in this regard. \\ 
%             Concern & How the application architecture can be improved or can be protected from unwanted rigidity by using a third partly library? \\ 
%             Solution & Use a library that just fits right with the application in terms of the size and flexibility. \\ 
%             Consider-ation & The library should be flexible to existing application and should not be too heavy in size compared to the required functionality. \\ 
%             Steps & Find, compare, and review libraries. In addition, conduct design review to assess impact on architecture. \\ 
%             Support & Review internal organizational content for design principles, and study study online articles for suggestions. \\ 
%             Example Trace in Data & the moment you have to bring something in because there are new requirements is the time to assess how you’ve structured your application and does it still serve you and your customers or the business requirements you have. So that’s the opportunity to look at the structural aspect of the application and make sure you do want to avoid changing it or maybe
%             it’s the time to change it. (P22) \\ 

%          \bottomrule
%     \end{tabular}
%     \label{tab:gp3-scenario}
% \end{table} 


%\begin{practice}{3}{Maximize Flexibility} % Was Improve Application Structure
\actor{Developers, Architects}
\condition{Large scale software applications can have their own structure which evolved over time and may follow some software design principles. System designers want to use new libraries in a way that improves the application structure, or at least does not deteriorate it.
When implementing critical features, architects often plan for extendability and maintainability. A third party application can pose both a risk and advantage in this regard.}
\concern{How can the application architecture  be improved or be protected from unwanted rigidity by using a third partly library?}
\solution{Use a library that just fits right with the application in terms of the size and flexibility. Consider wrapping the third-party library in an API to allow replacement of the library without affecting dependent code.}
\consideration{The library should be flexible and should not be too large in size compared to the required functionality.}
\steps{Find, compare, and review libraries. In addition, conduct design review to assess impact on architecture. Review internal organizational content for design principles, and study study online articles for suggestions.}
\example{The moment you have to bring something in because there are new requirements is the time to assess how you've structured your application and does it still serve you and your customers or the business requirements you have. So that’s the opportunity to look at the structural aspect of the application and make sure you do want to avoid changing it or maybe
it's the time to change it.}{P22}
\end{practice}

% \begin{table}[]
%     \centering
%     \caption{Scenario of GP3: Improve Application Structure}
%     \begin{tabular}{p{1cm}p{7.5cm}}
%          \toprule
%          \textbf{GP} & \textbf{Improve Application Structure} \\ 
%          \midrule
%             Actor(s) & Developers, Architects \\ 
%             Context & - Large scale software applications can have their own structure which evolved over time and may follow some software design principles. System designers would be using new libraries in a way that improve the application structure, or at least does deteriorate it.
%             - When implementing critical features, architects often plan about extendability and maintainability. A third party application can pose both a risk and advantage in this regard. \\ 
%             Concern & How the application architecture can be improved or can be protected from unwanted rigidity by using a third partly library? \\ 
%             Solution & Use a library that just fits right with the application in terms of the size and flexibility. \\ 
%             Consider-ation & The library should be flexible to existing application and should not be too heavy in size compared to the required functionality. \\ 
%             Steps & Find, compare, and review libraries. In addition, conduct design review to assess impact on architecture. \\ 
%             Support & Review internal organizational content for design principles, and study study online articles for suggestions. \\ 
%             Example Trace in Data & the moment you have to bring something in because there are new requirements is the time to assess how you’ve structured your application and does it still serve you and your customers or the business requirements you have. So that’s the opportunity to look at the structural aspect of the application and make sure you do want to avoid changing it or maybe
%             it’s the time to change it. (P22) \\ 

%          \bottomrule
%     \end{tabular}
%     \label{tab:gp3-scenario}
% \end{table}
%% GP 4
 %% Ann: I include this one because it is not something most people consider
\nd\bf{\ul{D4: Empower the Team.}} 
\begin{practice}{4}{Empower the Team}
\actor{Developers, Tech Leaders}
\condition{Some organizations may have a strong company culture to improve the development skill set or for providing comfortable learning space for developers. 
Tech leaders may care more for their team's capacity, limitations, and motivations.
The development team may have limitations or strengths in certain technologies. }
\concern{Does the library fit well with the capability of the development team? Will it provide them any transferable skills?}
\solution{Use a library that is appreciated by the developers}
\consideration{The library should be well documented and should have a popular community so that developers can easily adopt and can refer in future. Also it can have customer support in case developers needs extra help.}
\steps{Besides finding a library that fits well with the technology, thoroughly discuss with developers about their opinion and acceptance of the library. Look into official documentation of the library for documentation and support issues.}
\example{So looking at community popularity helps because then it helps to hire people. It helps to retain people. They like to use technologies that are transferable.}{P19}
\end{practice}


% \begin{table}[]
%     \centering
%     \caption{Scenario of GP4: Empower the Team}
%     \begin{tabular}{p{1cm}p{7.5cm}}
%          \toprule
%             \textbf{GP} & \textbf{Empower the Team} \\ 
%             \midrule
%             Actor(s) & Developers, Tech Leaders \\ 
%             Context & - Some organizations may have strong company culture to improve development skill set or for providing comfortable learning space for developers. 
%             - Tech leaders may care more for their team's capacity, limiatation, and motivation.
%             - Development team may have limitation or strength in certain technology  \\ 
%             Concern & Does the library fit well with the capability of the development team? Will it provide them any transferable skill? \\ 
%             Solution & Use a library that is appreciated by the developers \\ 
%             Consider-ation & Library should be well documented, should be community popular so that developers can easily adopt and can refer in future. Also it can have customer support in case developers needs extra help. \\ 
%             Steps & Besides finding a library that fits well with the technology, thoroughly discuss with developers about their opinion and acceptance of the library. \\ 
%             Support & Look into official documentation of the library for documentation and support issues. \\ 
%             Example Trace in Data & So looking at community popularity helps because then you can it helps to hire people. It helps to retain people. They like to use technologies that are transferable (P19) \\ 
%          \bottomrule
%     \end{tabular}
%     \label{tab:gp4-scenario}
% \end{table}
Adopting third-party libraries also empowers the development teams. Open-source libraries provide developers with working experience with popular development components to acquire transferable knowledge whereas proprietary solutions developed internally can create bottlenecks: 
\qi{When you move to use something that's an open source third-party solution, it's well documented, suddenly an entire team of people can help address a problem, and your speed to fixing something goes from multiple days or weeks to hours. And you empower the whole team rather than an [internal] specialist who can't go on vacation because if she does, sorry, you're waiting.}{P22}

 When following this empowering \principle, developers often consider \qqw{C}{peer opinions} during the library \qqw{P}{review} phase: 
  \qi{I'm not taking a decision for myself. Also, there are seniors in every organization who are more experienced and they may have some opinions on it.}{P01}
 \qi{Let's say, I proposed one library from my previous work. And then I have to explain it to the team members.}{P10}




%% GP 5
\nd\bf{\ul{D5: Ensure Compliance.}} Third-party libraries can come with licensing issues, privacy concerns, or security vulnerabilities.
However, not all development teams equally consider the impact of compliance issues. \qqw{C}{Legal environment} of the organization or their customers are the primary driver for compliance principle: 
\qi{Since we work with people from Europe, the UK, the US, and other areas and they have a very strong data security policy, we have to be confident that they're secure well.}{P04} 
\qi{So there are libraries who advertise that they are already GDPR \cite{website:gdpr} compliant or FedRAMP \cite{website:fedramp} compliant. So those kinds of libraries are always preferred over libraries that don't have a clear signal.}{P19} 

\begin{practice}{5}{Ensure Compliance}
\actor{Developers, Information Security Experts, Legal Experts, Open Source Program Office}
\condition{Matured, regulated org or industry (health, finance). Presence of dedicated security/legal experts.}
\concern{Any penalty or legal complication arising from using a library? How to protect the organization?}
\solution{Use a library which is compliant with the application security standards and legal requirements}
\consideration{License compatible with existing code and business. Secure, no known vulnerability.}
\steps{Reach out to specialists in the organization for taking their expert consent before adopting the library. See the license and security declarations in the library documentation in the source code or package repository.}
\consequence{Early stage companies may ignore compliance and can face existential crisis.}
\example{We had a very bad experience with this. With the legacy system, we were using so many different libraries and there is a licensing issue and we had to replace half of the library. Otherwise we had to pay lots of money. So that's why we are now very, very concerned about adding any external library, because if we don’t comply with the license, it will be a legal problem.}{P09}
\end{practice}

% \begin{table}[]
%     \centering
%     \caption{Scenario of GP5: Ensure Compliance}
%     \begin{tabular}{p{1cm}p{7.5cm}}
%          \toprule
%             \textbf{GP} & \textbf{Ensure Compliance} \\ 
%             \midrule
%             Actor(s) & Developers, Information Security Experts, Legal Experts, Open Source Program Office \\ 
%             Context & - Because of regulatory compliance or for company culture, some organizations will be more cautious about using third-party libraries for legal, security, and privacy reasons. 
%             - Developers will often have little technical expertise on such speialized issues
%             - Sometimes small or early stage companies may even ignore the importance of compliance issues
%             - Few application domains such as health, finance, media are also more regulated and require organizational policies for ensuring compliance. \\ 
%             Concern & Will there be any penalty or legal complication arising from using a third-party library? How to protect the organization? \\ 
%             Solution & Use a library which is compliant with the application security standards and legal requirements \\ 
%             Consider-ation & License of the library should be compatible with the business and license of the target software. The security and privacy concerns should be clarified and well take care of by the library contributors. \\ 
%             Steps & Reach out to specialists in the organization for taking their expert consent before adopting the library. \\ 
%             Support & See the license and security declarations in the library documentation in the source code or package repository. \\ 
%             Example Trace in Data & we had a very bad experience with this. With the legacy system, we were using so many different libraries and there is a licensing issue and we had to replace half of the library. Otherwise we had to pay lots of money. So that’s why we are now very, very concerned about adding any external library, because if we don’t comply with the license, it will be a legal problem. (P09) \\ 
%          \bottomrule
%     \end{tabular}
%     \label{tab:gp5-scenario}
% \end{table}
 %% Ann: I exclude this one because it is fairly evident (to save space)

%% GP 6
\nd\bf{\ul{D6: Maintain Continuous Stability.}} Aside from compliance issues, the biggest risk of using a third-party library is that the library won't be maintained.

\qi{If the contributors are gone in those third-party libraries repositories, suddenly the whole production application is broken}{P10} 

To protect themselves from abandonment and stability risks, developers look for libraries where a \qqw{F}{large community} and contributors are involved. To \qqw{P}{search information} about \qqw{F}{active development}, they check \qqw{S}{source repositories}: 
\qi{We have to check whether this library maintenance is active or not. We can determine that by checking their GitHub repository when the last push was given?}{P04} 

\begin{practice}{6}{Maintain Continuous Stability}
\actor{Developers, DevOps}
\condition{Long term application. Critical library update (vulnerability fix) or unmaintained library.}
\concern{How to ensure a library is maintained in foreseeable future and developers can use smoothly?}
\solution{Library with good history of maintenance and prepare to continuously upgrade the library in future}
\consideration{Actively maintained library, supported by reputed organizations, and has larger community.}
\steps{Analyze the maintenance and issue history of the library to assess the active development practices or the library. Establish a process for software bill of materials to document all third-party library dependencies and their upgrade plan in conjunction with DevOps teams. Look into source code commit and issue history from source repository and download usage trend from package repository.}
\consequence{Requires dedicated resource upgradation for smooth operation. Otherwise can break system.}
\example{When we integrated the updated version our whole interface broke. And we had to change a lot of code, all the interceptors, interfaces, everything\ldots This maintenance is quite hard. It's actually a full time work to always keep updated, to always stay updated.}{P14}
\end{practice}

% \begin{table}[]
%     \centering
%     \caption{Scenario of GP6: Maintain Continuous Stability}
%     \begin{tabular}{p{1cm}p{7.5cm}}
%          \toprule
%             \textbf{GP} & \textbf{Maintain Continuous Stability} \\
%             \midrule
%             Actor(s) & Developers, DevOps \\ 
%             Context & - Some software applications are developed and maintained for long term. A third-party library used in such a product can have bugs or vulnerabilities that need to be fixed. Sometimes, the contributors of the library may not continue to fix bugs or improve with new features. Sometimes libraries may not have backwards compatibility and when developers upgrade, their existing system can break. \\ 
%             Concern & How developers will ensure that a library is well maintained in foreseeable future and can keep using the library without breaking their application? \\ 
%             Solution & Use a library with good history of maintenance and prepare to continuously upgrade the library in future \\ 
%             Consider-ation & Selected libraries should be actively maintained by contributors, supported by reputed organizations, and have larger community. \\ 
%             Steps & Analyze the maintenance and issue history of the library to assess the active development practices or the library. Establish a process for software bill of materials to document all third-party library dependencies and their upgrade plan in conjunction with DevOps teams. \\ 
%             Support & Look into source code commit and issue history from source repository and download usage trend from package repository. \\ 
%             Example Trace in Data & when we integrated the updated version our whole interface broke. And we had to change a lot of code, all the interceptors, interfaces, everything... This maintenance is quite hard. It’s actually a full time work to always keep updated, to always stay updated. (P14) \\ 
%          \bottomrule
%     \end{tabular}
%     \label{tab:gp6-scenario}
% \end{table}
 %% Ann - excluded to save space


% During the theoretical integration of our study, the six guiding principles emerged from the benefits and baggage of libraries - why developers want to use a library and why they still remain cautious to use a library.













\section{Discussion}

\subsection{Recommendations}

We have observed that even before being able to integrate a library, there are certain conditions (C) in an organization or a developer which may prevent them from considering a library at all. Obstacles can be caused by organizational policy, technology, or even by the developer's mindset and experience. In some cases, a lack of supportive process, such as an Open Source Software Office (OSPO), or restricted internet access meant that developers lacked clarity about which libraries could be used. 
\begin{recommendation}{rec:ospo}
  {Organizations should formalize third-party library policy, and create streamlined, proportional processes to support it.}
\end{recommendation}\medskip

For developers and organizations the use of libraries not only brings benefits, it also brings disadvantages which must be considered. Participants noted that often developers can focus on solving imminent problems in hand and have a blind spot on the future maintenance of a library. However, as we have observed, library maintenance is a major part of the adoption process and developers should be aware of the \code{post integration maintenance} period. 
\begin{recommendation}{rec:maintenance}
  {Developers not only need to consider the maintenance related concerns while reviewing a library, they also need to deploy a strategy to continually maintain or upgrade libraries. The strategy may involve developing organizational policy, or even allocating certain resources for future maintenance.}
\end{recommendation}\medskip

Though we presented the conditions of each guiding principle separately, in reality, developers may have multiple types of conditions with conflicting guiding principles. For example, in a matured organization with large scale application, a team may face urgent production issue in a critical feature that may prompt 'Just Do It' principle immediately, however, they will need to adopt 'Reuse Robust Component' once the urgency is taken care of. %The long-term maintainability of a library is not always a consideration in all situations. While it is critical in the guiding principle (GP) ``Ensure Compliance,'' it is of little interest under the conditions where ``Just Do It'' is applicable. 
\begin{recommendation}{rec:gp}
%Developers should identify the appropriate guiding principle or principles before they can begin the library adoption steps.
Developers may need to balance among multiple guiding principles when faced with complicated \& conflicting conditions. 
\end{recommendation}\medskip

Marketing theories consider cut-off factors as those factors of product whose absence will bar the consumer from buying it \cite{blackwell2001consumer}. During our interviews, participants observed that all developers consider \code{capability} and \code{compatibility} of libraries as such cut-off factor. However, they also noted that developers in few conditions (small or \code{early stage} organizations, or \code{early career} developers) may be unaware of the severity of security and license issues and may ignore those. Industry experts strongly recommended to consider compliance issues as cut-off factors.
 \begin{recommendation}{rec:cutoff}
Organizations should consider \code{security} and \code{license} issues, irrespective of \code{organizational} or \code{environmental} conditions.
\end{recommendation}\medskip


Developers seek information from a variety of sources. If the organization does not have a very welcoming inclusive culture, critical analysis of libraries can be ultimately influenced and driven by outspoken people. We have heard from the interviewees about the necessity of inclusive culture where they assumed that the inclusivity should prevail in a software development team from the beginning of the recruitment process up to the development team's regular discussion. However, it was also evident that even in larger well managed organizations, such inclusivity may remain very subtle and may not be able to promote a democratic decision throughout the library selection process. It would be the responsibility of the team leader to let normally silent members communicate their ideas in whatever preferred (verbal, written) way possible.
 \begin{recommendation}{rec:inclusivity}
Organizations should cultivate openness and encourage inclusivity irrespective of individual's communication preference. 
\end{recommendation}\medskip

 Even when the culture promotes openness, development teams often have few enthusiastic developers who love to explore and whose opinions might have a disproportionate weight on discussions. Teams are better when all members will develop a habit of regular studies and in the long run they will be able to adapt much better with technological changes as well as make decisions about third-party tools or libraries in general. We found from the participants that enthusiastic early career developers often care about getting things done and frequently wants to try out new libraries for richness of their career (or resume). However, raising the concern of the lifelong maintenance of third-party libraries, interviewees recommended to avoid libraries when it is not necessary. 
  \begin{recommendation}{rec:hackathon}
To promote the culture of technical exploration and discussion, a team can decide to go for weekly studies and form a study circle where every developer will present something new that they learnt, or implement yearly hackathons where developers can experiment with new technologies without jeopardising the production system. 
\end{recommendation}


\subsection{Implications} 


Our library adoption model is similar to consumer and business buyer models of marketing theories \cite{kotler2014principles}. The library adoption conditions in our model match with business/organization buyer behavior conditions as the process involves intensive peer influences compared to consumer behavior. The two models are also similar in terms of actors. However, unlike business buying behavior model which has supplier search, selection, and performance review for traditional procurement process, the library adoption process is simpler and closely relates with the consumer buying behavior. 
%The consumer buying process has need recognition, information search, alternative evaluation, purchase decision, and post-purchase behavior. 
This is similar to the library adoption process, except that we have split the purchase decision to \code{review} (where mostly decision is taken) and \code{integrate} steps and merged \code{problem identification} under \code{search} step. The use of the marketing lens suggests future work in extending the Fishbein Multiattribute Model \cite{fishbein1967attitude} to library selection, potentially melding this area of research with the work which has already been done to identify processes for library selection (described in Section~\ref{sec:lit:processes}). Furthermore, our work offers the potential of creating a tool to support developer decisions which truely considers all important aspects of the selection process. In our later interviews, we presented developers with a mock-up of such a tool, the development of which is part of ongoing research.


Our research presents the software library adoption process for the first time, along with a detailed description of conditions, factors, and principles affecting the process. Unlike earlier work, which found 26 human and technical factors associated with adoption \cite{larios2020selecting}, we used a marketing lens to divide aspects of the selection process into different categories. We identified 14 new conditions (e.g., \code{company}, \code{application domain}, \code{investment capacity}) and 15 additional selection factors (e.g., \code{ease of installation}, \code{flexibility}, \code{availability of demo}, \code{familiarity}). 
% We were further able to identify eight novel information sources such as \code{known people}, \code{search engine}, \code{existing source code}) compared with an earlier study on collecting information about open source software \cite{li2022exploring}. 
We also present six guiding principles, which can be readily applied by industry. We supplement these with recommendations drawn from the interviews on how industry can more effectively prepare for and use third-party libraries. 

%As part of the adoption process, we have presented 5 major steps of library adoption process, identified 28 library selection factors, 14 library related source of information, and 23 internal external conditions that influence the process. No other previous study provided the process steps for libraries. Larios-Vargas et. \cite{larios2020selecting} identified 26 factors influence library selection under technical, human resource, and economical factors. Their proposed human resource and economical factors are fully mapped by our conditions category and technical factors are covered by our library selection factors. We identified 14 new conditions (such as \code{company}, \code{application domain}, \code{investment capacity}, \code{Expert/lead opinion}, \code{upgrades of OS}, \code{criticality of feature}, \code{delivery deadline}) beyond their human and economic factors. Moreover, our study also found 15 additional software selection factors (such as \code{ease of installation}, \code{flexibility}, \code{availability of demo}, \code{familiarity}, \code{used by reputed company}) which their study did not report. Besides, the factors, Li et. al outlined 14 sources for collecting information about open source software (not library) \cite{li2022exploring}. All of their reported sources are covered by our 6 information sources (\code{Package repository}, \code{Source repository}, \code{Stack Overflow}, \code{Other Q\&A sites}, \code{Online Article}, \code{Influential Blog}) and 8 of our sources are novel (such as \code{known people}, \code{search engine}, \code{existing source code}) for library information.}







\subsection{Quality and Applicability of the Study}
Corbin and Strauss did not recommend using the terms `validity and reliability' when discussing qualitative search \cite{corbin2014gt}, because qualitative methodologies cannot be assessed using quantitative criteria. They defined 17 measures researchers can use to evaluate the quality and applicability of their grounded theory research. We present the complete evaluation as part of our replication package \cite{website:replication-package}, here we present five major evaluation criteria.

\subsubsection{Industry Fitness}
This criterion concerns industry credibility. We conducted member checking \cite{creswell2016qualitative} by presenting a ten page summary of our findings, which was sent to 18 of our interviewees who agreed to further contact. This communication included a link to a survey which asked their opinion as to whether the summary reflected their experience and was useful to them. Thirteen participants completed the survey, all of whom opined that the summary was accurate. Five provided additional feedback on the utility of the findings:
\qi{I think the summary captures different aspects of library selection process very well. It outlines a generic, industry-wide pattern with enough details, and also provides exceptional factors that impact that pattern.}{}

\subsubsection{Industrial Application} Industrial application answers the question of whether the findings provide insight into situations and provide knowledge which can be applied to develop policy, change practice, and add to the knowledge base of a profession \cite{corbin2014gt}. This criterion was also evaluated through member checking. One participant shared their feedback, demonstrating that they found our work applicable: \qi{One interesting thing that I learned from your research is, different developers have different processes and priorities for picking a library, and not everybody is considering all the steps that need to be taken, so I would recommend your paper to all developers to just widen their horizons.}{}

\subsubsection{Usefulness} To address this measure, we looked at if there are suggestions for practice, policy, teaching, and application \cite{corbin2014gt}. To the best of our knowledge, this is the first study to provide a comprehensive picture of library adoption. The presentation of guiding principles in the form of patterns, which are widely used in industry, makes the results more accessible to practitioners. Participants also appreciated the suggestions:
\qi{My favorite part of the summary is the takeaway action items that I believe would be useful in building a better culture for adopting the right tools and technologies.}{}

\subsubsection{Explainability of Theory} The way to assess this measure is to determine if variation is built into the theory \cite{corbin2014gt}. Our conceptual framework is based on six guiding principles which depend on different contexts and lead to different implications for the influence of conditions, factors, and use of information sources. This allows the theory to support a wide variety of circumstances.

\subsubsection{Saturation of Categories} ``How is saturation explained, and when and how was it determined that categories were saturated?'' is the criterion for evaluating this measure \cite{corbin2014gt}. We provided detailed information about how we performed theoretical sampling to achieve saturation, and described how the major categories achieved saturation throughout the progression of the study. The Appendix contains a heatmap of concept saturation which shows the progression of topics across interviews.






\section{Conclusion}
Software library adoption includes the selection as well as the maintenance of a library. In this paper, we conducted \numInterviews semi-structured interview to explore the major steps developers follow in the adoption of a software library. We present a novel library adoption model that consists of a set of steps that developers follow to adopt a library, and a set of conditions, factors, and principles that influence the steps. We proposed six recommendations derived from the concerns that developers identified in interviews. Our study provides researchers with the opportunity to investigate specific adoption steps in more detail. The factors can be used to develop comparative analysis tools. Additionally, industry can make use of the principles and recommendations to guide decisions about third-party library selection. Our future work focuses on the development of a toolkit to support the automatic comparison of software libraries based on our derived library adoption model.
% As part of the adoption process, we presented five steps of the process, with 18 associated concepts. We showed how five conditions, consisting of 23 concepts, influence the choice of guiding principle, which in turn influences the weight of factors and the sources of information used to inform the adoption process. The four factors contain 28 concepts, while the five categories of information sources have 14 concepts associated with them.




% The conceptual framework of adoption process should provide the researchers the opportunity to investigate more into specific adoption steps. The factors can be used for developing comparative analysis tools for libraries. Researchers in the techno-social community can study the individual and organizational conditions that interplay in the library adoption process.

 
% The other major contribution of our research is the guiding principles. These principles not only explain the influences and activities in library adoption process, but also provide a concrete guideline for developers working in complex conditions to adopt libraries appropriately.







%%
%% The acknowledgments section is defined using the "acks" environment
%% (and NOT an unnumbered section). This ensures the proper
%% identification of the section in the article metadata, and the
%% consistent spelling of the heading.
\begin{acks}
\end{acks}

%%
%% The next two lines define the bibliography style to be used, and
%% the bibliography file.
\bibliographystyle{ACM-Reference-Format}
\bibliography{references}

%%
%% If your work has an appendix, this is the place to put it.
%\appendix
%%%%%%%% Appendix
\pagebreak
\section{Methodology}

Figure~\ref{fig:methodology} provides an overview of the overall research method which was applied to the study.

\begin{figure*}
    \centering
    \includegraphics[scale=.75]{images/methodology.pdf}
    \caption{Grounded theory research method applied}
    \label{fig:methodology}
\end{figure*}


\section{Saturation}
\begin{figure*}
    \centering
    \includegraphics[scale=0.9]{images/saturation.pdf}
    \caption{Heatmap of concept saturation over the interviews. Red refers to the higher discussion of a concept by an interviewee, the number refers to how many times the interviewee has discussed a concept. Green refers to the lower discussion of concepts by an interviewee. Thick black-bordered cells refer to the most discussion reference of a concept among all participants. For example, the review concept under the selection process category was discussed 26 times in P08's interview and hence the corresponding cell is marked by the thick border. The last two rows show the number of concept saturation by each participant and the cumulative number of concepts saturated till that participant. For example, the interview with P08 alone saturated 3 concepts (denoted by the thick boxes in the P08 column) and after the P08 interview, total of 17 concepts were saturated.}
    \label{fig:saturation}
\end{figure*}
Figure~\ref{fig:saturation} shows how the concepts were discussed during each interview. The number denote how many times a concept was discussed by one particular interview. The more a participant discussed about a particular concept, the more red the corresponding cell is. For example, library search and analysis process was most discussed by P5 and after their interview, subsequently we did not have much to discuss about the search process.  After P8, the concept almost saturated and we discussed very little about this concept with subsequent interviewees.

\begin{table*}[]
    \centering
    \caption{How we recruited interview participants following theoretical sampling for Concept Saturation. (We could not enhance targeted concepts from the *-marked participants (P12, P16), rather enriched other important concepts.)}
    % \renewcommand{\arraystretch}{1.5} \minaoar{if we add spacing, this table does not fit in a page.}
    \begin{tabular}{>{\raggedright}p{.4cm}p{3.8cm}p{5.5cm}p{5cm}}%{llll}
    \toprule
    P\# & Concept we wanted to enhance & Why we selected this Participant & Concepts they enriched significantly \\ 
    \midrule
P1 & Initial process and factors & Architect of a large system & Library definition, factors, influences \\ 
P2 & Licensing and Security Issues & Working in a large structured company & License, company technology \\ 
P3 & Mobile development Factors & 12+ years experienced in mobile application & Cost, company tech, comparison  \\ 
P4 & Long term maintenance concerns & Being a CEO, takes decisions considering long term impact & Company application domain, active development of library \\ 
P5 & Decision making processes & Stablishing the processes in a startup team & Information search, company culture \\ 
P6 & Open Source factors & Has experience regarding OSS contribution and research & Open source, Personal motivation \\ 
P7 & Factors for a startup & Being a startup CTO may share different priorities & Flexibility, Ease of Installation, Community Support \\ 
P8 & Performance factors & Working in a cloud company that may require high performing libraries & Familiarity, Team Discussion, Library Migration \\ 
P9 & Migration scenarios & Experienced to migrate company tech stack as architect & Legal risks, Lack of Stability, Less prefered than native support \\ 
P10 & Visualization and front end libraries & Working as web developer for over a decade & Customer support, flexibility, existing repository \\ 
P11 & Machine learning libraries & Experienced in machine learning in gradudate research studies and in industry & Talk to people, Performance, Outstanding library selection \\ 
P12* & DevOps Process for Library Security Issues & Consulted dozens of companies in DevOps process establishment & Barriers of library usage, Baggage of libraries \\ 
P13 & Selection process in large organizations for legal and security risks & Has been an architect in a large team for 10+ years & Consent Process, Benefits of libraries, Tech Expert Opinion \\ 
P14 & Library migration scenarios & Experienced in managing mobile apps with large user base in all platforms & Make life easy, Life long maintenance, Migration to other library \\ 
P15 & Organizational process and motivation for libraries & Experienced in organization process since increased dev team from 3 to 300 & Delivery Deadline, Don't Reinvent the wheel, Feature criticality \\ 
P16* & Process of security concerns & Cerified security professional actively developing security products & License issues, Data Transfer Security, Geographic Impact  \\ 
P17 & Security Process & Delivers custom software to customers and maintains SecOps in CI/CD & Post Integration Maintenance for Security \\ 
P18 & C++ libraries in large scale long term products & Leads development of a 30 year old product written in C++ with 2M lines of code & Lifelong Maintenance Burden, Compatibility, Uniform Coding Style \\ 
P19 & Company Culture, Open Source, Concept Saturation & Experienced working in start-up and large organizations who open source libraries & Standard practices in large organizations, Considerations in open source \\ 
P20 & Challenges in mobile application libraries, Concept Saturation & Full career in mobile app development, mostly in iOS which requires more maintenance & Lifelong Maintenance Burden, Abandoned Libraries, Migration \\ 
P21 & Company Culture, Open Source, Concept Saturation & Works full-time in a prominent open core company & Company policies, Guiding Principles \\ 
P22 & Guiding Principles, Open Source & Experienced in persuing large corporation for open source library adoption & Guiding Principles \\ 
P23 & ML libraries & Working in South America in ML domain & ML Library Dependency Issues \\ 
P24 & Company Culture, Industry, ML Libraries & Working in health sector using ML libraries extensively. & ML Library deployment and upgrade issues \\ 

\bottomrule
    \end{tabular}
    \label{tab:theoritcal-sampling}
\end{table*}


%% Ann: this is just an alternate view of our main figure and I don't think it is useful
%\begin{figure}
%    \centering
%    \includegraphics[scale=0.6]{images/Interaction-with-guiding-principles-5.png}
%    \caption{Interaction of guiding principles with the major concepts of software library adoption process}
  %  \label{fig:gp-interaction}
%\end{figure}


\section{Code System}

Figures~\ref{fig:process} and \ref{fig:conditions} show the relationship between major categories, categories, and concepts in the code system. These concepts and categories contribute to the conceptual framework of the software library adoption process.

\begin{figure*}
    \centering
    \includegraphics[scale=0.85]{images/process.pdf}
    \caption{Concepts related library adoption process}
    \label{fig:process}
\end{figure*}

\begin{figure*}
    \centering
    \includegraphics[scale=0.85]{images/conditions.pdf}
    \caption{Concepts related with factors and conditions influencing library adoption process}
    \label{fig:conditions}
\end{figure*}

\section{Guiding Principles}
In this section, we present the six full patterns associated with the library adoption steps.

\FloatBarrier
\begin{practice}{1}{Just Do It}
\actor{Developers}
\condition{Faster go to market is critical for business. Developers are often rewarded for delivering minimum viable product on time.}
\concern{How the developers can meet the deadline with relatively less effort?}
\solution{Use a third-party library that reduces the work load and delivery can be done on time}
\consideration{The library should be easy to use, easy to install, popular, and even better if developer is already familiar with it.}
\steps{Find libraries, compare them, and choose one according to the consideration. Take support from known people or use search engine or Stack Overflow}
\example{For startups, a lot of it [priority] is just speed to market and how much resources is gonna eat up using any specific library.}{P07}
\end{practice}

% \begin{table}[]
%     \centering
%     \caption{Scenario of GP1: Just Do It}
%     \begin{tabular}{p{1cm}p{6.5cm}}
%          \toprule
%             \textbf{GP} & \textbf{Just Do It} \\ 
%             \midrule
%             Actor(s) & Developers \\ 
%             Context & - Faster go to market is critical for business. Developers are often rewarded for delivering minimum viable product on time.
%             - Developers are in their early career and assume higher satisfaction on faster delivery. \\ 
%             Concern & How the developers can meet the deadline in relatively less effort? \\ 
%             Solution & Use a third-party library that reduces the work load and delivery can be done on time \\ 
%             Consider-ation & The library should be easy to use, easy to install, popular, and even better if developer is already familiar with it. \\ 
%             Steps & Find libraries, compare them, and choose one according to the consideration \\ 
%             Support & Take support from known people to know about such library or use search engine, Stack Overflow \\ 
%             Example Trace in Data & For startups, a lot of it [priority] is just speed to market and how much resources is gonna eat up using any specific library. (P07) \\ 
%          \bottomrule
%     \end{tabular}
%     \label{tab:gp1-scenario}
% \end{table}

\begin{practice}{2}{Reuse Robust Component}
\actor{Developers, Senior Developers}
\condition{Mature organization with stable source code and release pipeline. Application performance and maintainability is a big concern for the team.
Developers with higher experience are careful about the quality of the application source code.}
\concern{How can the application avoid boiler plate code and follow best design principles?}
\solution{Use a trusted proven third party library that will keep the code clean and manageable.}
\consideration{The library should be open source, trusted by community, stable and provide apppropriate performance metrics required}
\steps{Find and compare libraries, review thoroughly by more than one developer. Look into the library's source code repository to analyze the stability, quality of the library, and also consider reputed technical blogs.}
\example{So [this large corporation] as a whole is actually built on the open source libraries that are suitable for our use cases. But that actually has been one of our primary focus as well. If you find a library, use it; only build if you can’t find anything.}{P13}
\end{practice}

% \begin{table}[]
%     \centering
%     \caption{Scenario of GP2: Reuse Robust Component}
%     \begin{tabular}{p{1cm}p{7.5cm}}
%          \toprule
%             \textbf{GP} & \textbf{Reuse Robust Component} \\ 
%             \midrule
%             Actor(s) & Developers, Senior Developers \\ 
%             Context & - Matured organization with stable source code and release pipeline. Application performance and maintainability is a big concern for the team.
%             - Developers with higher experience would be careful about the quality of the application source code. \\ 
%             Concern & How the application can avoid boiler plate code and follow best design principles? \\ 
%             Solution & Use a trusted proven third party library that will keep the code clean and manageable. \\ 
%             Consider-ation & The library should be open source, trusted by community, stable and provide apppropriate performance metrics required \\ 
%             Steps & Find and compare libraries, review thoroughly by more than one developer. \\ 
%             Support & Look into the library's source code repository to analyze the stability, quality of the library, and also consider reputed technical blogs. \\ 
%             Example Trace in Data & So [this large corporation] as a whole is actually built on the open source libraries that are suitable for our use cases. But that actually has been one of our primary focus as well. If you find a library, use it; only build if you can’t find anything. (P13) \\ 
%          \bottomrule
%     \end{tabular}
%     \label{tab:gp2-scenario}
% \end{table}
\begin{practice}{3}{Maximize Flexibility} % Was Improve Application Structure
\actor{Developers, Architects}
\condition{Large scale software applications can have their own structure which evolved over time and may follow some software design principles. System designers want to use new libraries in a way that improves the application structure, or at least does not deteriorate it.
When implementing critical features, architects often plan for extendability and maintainability. A third party application can pose both a risk and advantage in this regard.}
\concern{How can the application architecture  be improved or be protected from unwanted rigidity by using a third partly library?}
\solution{Use a library that just fits right with the application in terms of the size and flexibility. Consider wrapping the third-party library in an API to allow replacement of the library without affecting dependent code.}
\consideration{The library should be flexible and should not be too large in size compared to the required functionality.}
\steps{Find, compare, and review libraries. In addition, conduct design review to assess impact on architecture. Review internal organizational content for design principles, and study study online articles for suggestions.}
\example{The moment you have to bring something in because there are new requirements is the time to assess how you've structured your application and does it still serve you and your customers or the business requirements you have. So that’s the opportunity to look at the structural aspect of the application and make sure you do want to avoid changing it or maybe
it's the time to change it.}{P22}
\end{practice}

% \begin{table}[]
%     \centering
%     \caption{Scenario of GP3: Improve Application Structure}
%     \begin{tabular}{p{1cm}p{7.5cm}}
%          \toprule
%          \textbf{GP} & \textbf{Improve Application Structure} \\ 
%          \midrule
%             Actor(s) & Developers, Architects \\ 
%             Context & - Large scale software applications can have their own structure which evolved over time and may follow some software design principles. System designers would be using new libraries in a way that improve the application structure, or at least does deteriorate it.
%             - When implementing critical features, architects often plan about extendability and maintainability. A third party application can pose both a risk and advantage in this regard. \\ 
%             Concern & How the application architecture can be improved or can be protected from unwanted rigidity by using a third partly library? \\ 
%             Solution & Use a library that just fits right with the application in terms of the size and flexibility. \\ 
%             Consider-ation & The library should be flexible to existing application and should not be too heavy in size compared to the required functionality. \\ 
%             Steps & Find, compare, and review libraries. In addition, conduct design review to assess impact on architecture. \\ 
%             Support & Review internal organizational content for design principles, and study study online articles for suggestions. \\ 
%             Example Trace in Data & the moment you have to bring something in because there are new requirements is the time to assess how you’ve structured your application and does it still serve you and your customers or the business requirements you have. So that’s the opportunity to look at the structural aspect of the application and make sure you do want to avoid changing it or maybe
%             it’s the time to change it. (P22) \\ 

%          \bottomrule
%     \end{tabular}
%     \label{tab:gp3-scenario}
% \end{table}
\begin{practice}{4}{Empower the Team}
\actor{Developers, Tech Leaders}
\condition{Some organizations may have a strong company culture to improve the development skill set or for providing comfortable learning space for developers. 
Tech leaders may care more for their team's capacity, limitations, and motivations.
The development team may have limitations or strengths in certain technologies. }
\concern{Does the library fit well with the capability of the development team? Will it provide them any transferable skills?}
\solution{Use a library that is appreciated by the developers}
\consideration{The library should be well documented and should have a popular community so that developers can easily adopt and can refer in future. Also it can have customer support in case developers needs extra help.}
\steps{Besides finding a library that fits well with the technology, thoroughly discuss with developers about their opinion and acceptance of the library. Look into official documentation of the library for documentation and support issues.}
\example{So looking at community popularity helps because then it helps to hire people. It helps to retain people. They like to use technologies that are transferable.}{P19}
\end{practice}


% \begin{table}[]
%     \centering
%     \caption{Scenario of GP4: Empower the Team}
%     \begin{tabular}{p{1cm}p{7.5cm}}
%          \toprule
%             \textbf{GP} & \textbf{Empower the Team} \\ 
%             \midrule
%             Actor(s) & Developers, Tech Leaders \\ 
%             Context & - Some organizations may have strong company culture to improve development skill set or for providing comfortable learning space for developers. 
%             - Tech leaders may care more for their team's capacity, limiatation, and motivation.
%             - Development team may have limitation or strength in certain technology  \\ 
%             Concern & Does the library fit well with the capability of the development team? Will it provide them any transferable skill? \\ 
%             Solution & Use a library that is appreciated by the developers \\ 
%             Consider-ation & Library should be well documented, should be community popular so that developers can easily adopt and can refer in future. Also it can have customer support in case developers needs extra help. \\ 
%             Steps & Besides finding a library that fits well with the technology, thoroughly discuss with developers about their opinion and acceptance of the library. \\ 
%             Support & Look into official documentation of the library for documentation and support issues. \\ 
%             Example Trace in Data & So looking at community popularity helps because then you can it helps to hire people. It helps to retain people. They like to use technologies that are transferable (P19) \\ 
%          \bottomrule
%     \end{tabular}
%     \label{tab:gp4-scenario}
% \end{table}
\begin{practice}{5}{Ensure Compliance}
\actor{Developers, Information Security Experts, Legal Experts, Open Source Program Office}
\condition{Matured, regulated org or industry (health, finance). Presence of dedicated security/legal experts.}
\concern{Any penalty or legal complication arising from using a library? How to protect the organization?}
\solution{Use a library which is compliant with the application security standards and legal requirements}
\consideration{License compatible with existing code and business. Secure, no known vulnerability.}
\steps{Reach out to specialists in the organization for taking their expert consent before adopting the library. See the license and security declarations in the library documentation in the source code or package repository.}
\consequence{Early stage companies may ignore compliance and can face existential crisis.}
\example{We had a very bad experience with this. With the legacy system, we were using so many different libraries and there is a licensing issue and we had to replace half of the library. Otherwise we had to pay lots of money. So that's why we are now very, very concerned about adding any external library, because if we don’t comply with the license, it will be a legal problem.}{P09}
\end{practice}

% \begin{table}[]
%     \centering
%     \caption{Scenario of GP5: Ensure Compliance}
%     \begin{tabular}{p{1cm}p{7.5cm}}
%          \toprule
%             \textbf{GP} & \textbf{Ensure Compliance} \\ 
%             \midrule
%             Actor(s) & Developers, Information Security Experts, Legal Experts, Open Source Program Office \\ 
%             Context & - Because of regulatory compliance or for company culture, some organizations will be more cautious about using third-party libraries for legal, security, and privacy reasons. 
%             - Developers will often have little technical expertise on such speialized issues
%             - Sometimes small or early stage companies may even ignore the importance of compliance issues
%             - Few application domains such as health, finance, media are also more regulated and require organizational policies for ensuring compliance. \\ 
%             Concern & Will there be any penalty or legal complication arising from using a third-party library? How to protect the organization? \\ 
%             Solution & Use a library which is compliant with the application security standards and legal requirements \\ 
%             Consider-ation & License of the library should be compatible with the business and license of the target software. The security and privacy concerns should be clarified and well take care of by the library contributors. \\ 
%             Steps & Reach out to specialists in the organization for taking their expert consent before adopting the library. \\ 
%             Support & See the license and security declarations in the library documentation in the source code or package repository. \\ 
%             Example Trace in Data & we had a very bad experience with this. With the legacy system, we were using so many different libraries and there is a licensing issue and we had to replace half of the library. Otherwise we had to pay lots of money. So that’s why we are now very, very concerned about adding any external library, because if we don’t comply with the license, it will be a legal problem. (P09) \\ 
%          \bottomrule
%     \end{tabular}
%     \label{tab:gp5-scenario}
% \end{table}

\begin{practice}{6}{Maintain Continuous Stability}
\actor{Developers, DevOps}
\condition{Long term application. Critical library update (vulnerability fix) or unmaintained library.}
\concern{How to ensure a library is maintained in foreseeable future and developers can use smoothly?}
\solution{Library with good history of maintenance and prepare to continuously upgrade the library in future}
\consideration{Actively maintained library, supported by reputed organizations, and has larger community.}
\steps{Analyze the maintenance and issue history of the library to assess the active development practices or the library. Establish a process for software bill of materials to document all third-party library dependencies and their upgrade plan in conjunction with DevOps teams. Look into source code commit and issue history from source repository and download usage trend from package repository.}
\consequence{Requires dedicated resource upgradation for smooth operation. Otherwise can break system.}
\example{When we integrated the updated version our whole interface broke. And we had to change a lot of code, all the interceptors, interfaces, everything\ldots This maintenance is quite hard. It's actually a full time work to always keep updated, to always stay updated.}{P14}
\end{practice}

% \begin{table}[]
%     \centering
%     \caption{Scenario of GP6: Maintain Continuous Stability}
%     \begin{tabular}{p{1cm}p{7.5cm}}
%          \toprule
%             \textbf{GP} & \textbf{Maintain Continuous Stability} \\
%             \midrule
%             Actor(s) & Developers, DevOps \\ 
%             Context & - Some software applications are developed and maintained for long term. A third-party library used in such a product can have bugs or vulnerabilities that need to be fixed. Sometimes, the contributors of the library may not continue to fix bugs or improve with new features. Sometimes libraries may not have backwards compatibility and when developers upgrade, their existing system can break. \\ 
%             Concern & How developers will ensure that a library is well maintained in foreseeable future and can keep using the library without breaking their application? \\ 
%             Solution & Use a library with good history of maintenance and prepare to continuously upgrade the library in future \\ 
%             Consider-ation & Selected libraries should be actively maintained by contributors, supported by reputed organizations, and have larger community. \\ 
%             Steps & Analyze the maintenance and issue history of the library to assess the active development practices or the library. Establish a process for software bill of materials to document all third-party library dependencies and their upgrade plan in conjunction with DevOps teams. \\ 
%             Support & Look into source code commit and issue history from source repository and download usage trend from package repository. \\ 
%             Example Trace in Data & when we integrated the updated version our whole interface broke. And we had to change a lot of code, all the interceptors, interfaces, everything... This maintenance is quite hard. It’s actually a full time work to always keep updated, to always stay updated. (P14) \\ 
%          \bottomrule
%     \end{tabular}
%     \label{tab:gp6-scenario}
% \end{table}


\end{document}
