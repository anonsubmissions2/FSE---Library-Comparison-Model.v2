
\section{Conclusion}

We conducted \numInterviews semi-structured interviews using a sampling strategy based on concept saturation to explore the major steps developers follow in the adoption of a software library (RQ1), the conditions and technical factors influencing the process (RQ2), and the guiding principles of library adoption (RQ3).

Our main contributions were identifying a conceptual framework based on a library adoption life cycle and governed by six guiding principles. Using a marketing lens, we distinguished between conditions, factors, and sources affecting the adoption process. In the process we noted that using libraries comes with both benefits and disadvantages. We proposed six recommendations derived from the concerns that developers identified in interviews, which have the potential to improve how companies both prepare to and eventually use third-party libraries.

As part of the adoption life cycle, we presented five steps of the life cycle, with 18 associated concepts. We showed how five conditions, consisting of 23 concepts, influence the choice of guiding principle, which in turn influences the weight of factors and the sources of information used to inform the adoption process. The four factors contain 28 concepts, while the five categories of information sources have 14 concepts associated with them.

Our study provides researchers with the opportunity to investigate specific adoption steps in more detail. The factors can be used to develop comparative analysis tools. Additionally, industry can make use of the guiding principles and recommendations to guide decisions about third-party library selection.



% The conceptual framework of adoption life cycle should provide the researchers the opportunity to investigate more into specific adoption steps. The factors can be used for developing comparative analysis tools for libraries. Researchers in the techno-social community can study the individual and organizational conditions that interplay in the library adoption life cycle.

 
% The other major contribution of our research is the guiding principles. These principles not only explain the influences and activities in library adoption life cycle, but also provide a concrete guideline for developers working in complex conditions to adopt libraries appropriately.


