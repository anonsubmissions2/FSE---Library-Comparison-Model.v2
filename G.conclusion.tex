
\section{Conclusion}
Software library adoption includes the selection as well as the maintenance of a library. In this paper, we conducted \numInterviews semi-structured interview to explore the major steps developers follow in the adoption of a software library. We present a novel library adoption model that consists of a set of steps that developers follow to adopt a library, and a set of conditions, factors, and principles that influence the steps. We proposed six recommendations derived from the concerns that developers identified in interviews. Our study provides researchers with the opportunity to investigate specific adoption steps in more detail. The factors can be used to develop comparative analysis tools. Additionally, industry can make use of the principles and recommendations to guide decisions about third-party library selection. Our future work focuses on the development of a toolkit to support the automatic comparison of software libraries based on our derived library adoption model.
% As part of the adoption process, we presented five steps of the process, with 18 associated concepts. We showed how five conditions, consisting of 23 concepts, influence the choice of guiding principle, which in turn influences the weight of factors and the sources of information used to inform the adoption process. The four factors contain 28 concepts, while the five categories of information sources have 14 concepts associated with them.




% The conceptual framework of adoption process should provide the researchers the opportunity to investigate more into specific adoption steps. The factors can be used for developing comparative analysis tools for libraries. Researchers in the techno-social community can study the individual and organizational conditions that interplay in the library adoption process.

 
% The other major contribution of our research is the guiding principles. These principles not only explain the influences and activities in library adoption process, but also provide a concrete guideline for developers working in complex conditions to adopt libraries appropriately.


